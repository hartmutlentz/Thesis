\KOMAoptions{numbers=noenddot}
\usepackage{amsmath,amssymb,amsfonts,amsthm,epigraph,scrpage2}
\usepackage[ngerman,english]{babel}
\definecolor{Cayenne}{rgb}{0.502,0.0,0.0}
\definecolor{Steel}{rgb}{0.4,0.4,0.4}


%\setcounter{secnumdepth}{3} % sub subsections numbering
%\setcounter{tocdepth}{3} % subsubsections inTOC

\usepackage[format=plain,singlelinecheck=false, font={sf,small},labelfont={bf,color=Steel}]{caption}
\DeclareCaptionLabelSeparator{cayenne_period}{\textcolor{Cayenne}{.} }
\captionsetup{labelsep=cayenne_period}

% Colors
\addtokomafont{chapter}{\color{Steel}}
\addtokomafont{section}{\color{Steel}}
\addtokomafont{subsection}{\color{Steel}}
\addtokomafont{subsubsection}{\color{Steel}}
\addtokomafont{paragraph}{\color{Steel}}

\addtokomafont{pagehead}{\color{Steel}}
\renewcommand{\pnumfont}{\color{Steel}} 
\addtokomafont{headsepline}{\color{Steel}} 
\pagestyle{scrheadings} 

%\makeatletter % dot after sections and all below
%\let\std@sect\@sect
%\def\@sect#1#2#3#4#5#6[#7]#8{\std@sect{#1}{#2}{#3}{#4}{#5}{#6}[#7.]{#8\color{Cayenne}{.}}}
%\makeatother
\usepackage{acronym}

\usepackage[leftcaption]{sidecap} % inner, outer,left,right
\sidecaptionvpos{figure}{t}

% Papiergr��e
%\setlength{\paperwidth}{21cm}
%\setlength{\paperheight}{25cm}
%\recalctypearea
%\usepackage{geometry}
%\usepackage[cross,a4,center]{crop}

%% Flattersatz
%\usepackage[document]{ragged2e} % Flattersatz
%\setlength{\RaggedRightParindent}{1em} % evtl. parskip


%% Sans Serif
%\usepackage{cmbright}
%\renewcommand{\familydefault}{\sfdefault}
%% Palatino
%\usepackage[sc]{mathpazo}
%\linespread{1.05}         % Palatino needs more leading (space between lines)
%\setkomafont{sectioning}{\normalcolor\bfseries} % Kapitel�berschriften

%%% Kapitel�berschriften: Mit gro�en Zahlen
%\usepackage{titlesec}
%\titleformat{\chapter}[display]
%{\bfseries\Large}
%{ %\Huge\textsc{\chaptertitlename} % f�r das Wort 'Kapitel'
%\hfill\fontsize{120}{70}\selectfont\color{lightgray}\textbf{\thechapter}}
%{-2ex}
%%{\filleft\fontsize{50}{70}\selectfont\scshape} % Kapit�lchen oder...
%{\filleft\fontsize{50}{70}\selectfont\textbf} % ...oder keine Kapit�lchen
%[\vspace{0ex}]
%
%%%% Part�berschriften
%\titleformat{\part}[display]
%{\bfseries\Large}
%{ %\Huge\textsc{\chaptertitlename} % f�r das Wort 'Kapitel'
%\hfill\fontsize{120}{70}\selectfont\color{lightgray}\textbf{\thepart}}
%{-2ex}
%{\filleft\fontsize{50}{70}\selectfont\scshape} % Kapit�lchen oder...
%%{\filleft\fontsize{50}{70}\selectfont\textbf} % ...oder keine Kapit�lchen
%[\vspace{0ex}]


\newcommand{\ER}{Erd\H{o}s-R\'enyi }
\newcommand{\BA}{Barab\'asi-Albert }
\newcommand{\mean}[1]{\left< #1 \right>}
\newcommand{\abs}[1]{\left| #1 \right|}
\newcommand{\norm}[1]{\lVert#1\rVert}
\newcommand{\mat}[1]{\mathbf{#1}}
\newcommand{\tgraph}{\mathcal{G}}

\theoremstyle{definition} % non-italic
\newtheorem{annahme}{Annahme} % braucht amsthm
\newtheorem{definition}{Definition}
\newtheorem{theorem}{Theorem}
\newtheorem{satz}{Satz}
\newtheorem{frage}{Frage}
%\input{watermarks/watermark.tex}
\DeclareMathOperator{\nnz}{nnz}

% + Graphicspath nach begin document
