% Template to be used while publishing    
% a scientific work (PhD Thesis etc.)
% at the EDOC-Server of HU-Berlin                      
%    developed 2003-2008 by
% AG Elektronisches Publizieren, 
% Computer- und Medienservice,
% Humboldt-Universitaet zu Berlin
%    with friendly support of
% TeX-Stammtisch in Berlin     

% $Revision: 19 $
% $HeadURL: svn+ssh://ryckojox@svn.cms.hu-berlin.de/svn/projects/epub/latex/hudiss/mustermann.tex $
% $Date: 2009-07-03 15:49:19 +0200 (ptk, 03 lip 2009) $
% $Author$
% $Id: mustermann.tex 19 2009-07-03 13:49:19Z ryckojox $
                             
% Questions, comments, support:
%    edoc-latex@cms.hu-berlin.de    

% Documentation and information about the conditions of a publication:
%    http://edoc.hu-berlin.de/e_autoren/latex          

% Upload:
%    https://edoc.hu-berlin.de/cgi/dokupload/dokupload.cgi

\documentclass[openright,twoside,headsepline,% 
]{scrbook}[2007/12/24]

% The following package is necessary.
% To change the default options of the packages, use the key=value interface.
% See the documentation for details.
% If you need to use any special characters or LaTeX-commands 
% within the options, use \hudisssetup{} after loading this package,
% otherwise they will NOT work correctly.
\usepackage[
    inputenc=applemac, % default - latin1
    %fontset=palatino, % other possible parameters: lmodern, times, palatino
    natbib={numbers,sort}, %include to use the natbib-package
    % jurabib={}, %include to use the jurabib-package
    % apacite={}, %include to use the apacite-package
    hints=true, % set to 'false' for submission
    checktabu=False, % set to 'false' for submission
    draft=False, % set to 'false' for submission
    qserie=false
 ]{hudiss}

%\usepackage{layouts} % nur f�r Ausgabe der Textbreite
% Breite hier: B=14,6979 cm. Ergibt H=8,3988 cm f�r 7/4 Verh�ltnis

                
\hudisssetup{%
    titlepagefont={\Large\sffamily} % Use to change the titlepage font
}

% Fill out all the metadata here:
\hudissmetadata{%
    authorprefix={Dipl.-Phys.}, % e.g. Dipl.-Inf.
    authorfirstname={Hartmut}, % first name
    authorsurname={Lentz}, % surname
    authorsuffix={}, % e.g. Ph.D.
    authoradd={10/23/1980, Siegen}, % date and place of birth
    doctitle={Complex Networks}, % title of the thesis
    docsubtitle={for one euro}, % subtitle of the thesis
    % docsubject={}, % subject of the thesis (used in the properties of the pdf-document)
    approvala={Prof. Dr. Sokolov}, % approvals: a-e
    approvalb={Prof. Dr. Schimanski-Geier},
    approvalc={Prof. Dr. Selhorst},
    % approvald={},
    % approvale={},
    degree={Dr. Rer. Nat.}, % e.g. Dr. Rer. Nat.
    subject={Physik}, % e.g. Informatik
    faculty={Mathematisch-Wissenschaftlichen Fakult\"at I}, % in Dativ/Genitiv! e.g. Mathematisch-Wissenschaftlichen Fakult\"at II
    university={Humboldt-Universit\"at zu Berlin}, % e.g. Humboldt-Universit\"at zu Berlin
    dean={Prof. Stefan Hecht PhD}, % dean of the faculty
    president={Prof. Dr. Jan-Hendrik Olbertz}, % president of the university
    % datesubmitted={}, % the date of the submission
    % dateexam={}, % the date of your last exam
    % keywordsen={}, % english keywords comma separated
    % keywordsde={} % german keywords comma separated
}

% If you wish to load any further packages, 
% make any own adjustments (e. g. for the package fancyhdr)
% or define any own commands
% put ALL of them in the following file:
\KOMAoptions{numbers=noenddot}
\usepackage{amsmath,amssymb,amsfonts,amsthm,epigraph,scrpage2}
\usepackage[ngerman,english]{babel}
\usepackage{centernot} % f�r Durchstreichung

\usepackage{currvita}
\renewcommand*{\cvheadingfont}{\large\bfseries} % CV Titel
\renewcommand*{\cvlistheadingfont}{\bfseries\sffamily} % sub-�berschriften
\renewcommand*{\cvlabelfont}{\sffamily} % items

\definecolor{Cayenne}{rgb}{0.502,0.0,0.0}
\definecolor{Steel}{rgb}{0.4,0.4,0.4}
\definecolor{Tri_blue}{rgb}{0.039,0.5098,0.8}
\definecolor{Tri_yellow}{rgb}{0.5529,0.5451,0.0549}


%\setcounter{secnumdepth}{3} % sub subsections numbering
%\setcounter{tocdepth}{3} % subsubsections inTOC

\usepackage[format=plain,singlelinecheck=false, font={sf,small},labelfont={bf,color=Steel}]{caption}
\DeclareCaptionLabelSeparator{cayenne_period}{\textcolor{Cayenne}{.} }
\captionsetup{labelsep=cayenne_period}

% Colors
\addtokomafont{chapter}{\color{Steel}}
\addtokomafont{section}{\color{Steel}}
\addtokomafont{subsection}{\color{Steel}}
\addtokomafont{subsubsection}{\color{Steel}}
\addtokomafont{paragraph}{\color{Steel}}

\addtokomafont{pagehead}{\color{Steel}}
\renewcommand{\pnumfont}{\color{Steel}} 
\addtokomafont{headsepline}{\color{Steel}} 
\pagestyle{scrheadings}

%  labels in description environments
\renewcommand{\descriptionlabel}{\hspace\labelsep{}\sffamily\small\bfseries{}\color{Steel}{}}

%\makeatletter % dot after sections and all below
%\let\std@sect\@sect
%\def\@sect#1#2#3#4#5#6[#7]#8{\std@sect{#1}{#2}{#3}{#4}{#5}{#6}[#7.]{#8\color{Cayenne}{.}}}
%\makeatother
\usepackage{acronym}

\usepackage[leftcaption]{sidecap} % inner, outer,left,right
\sidecaptionvpos{figure}{t}

% Papiergr��e
%\setlength{\paperwidth}{21cm}
%\setlength{\paperheight}{25cm}
%\recalctypearea
%\usepackage{geometry}
%\usepackage[cross,a4,center]{crop}

%% Flattersatz
%\usepackage[document]{ragged2e} % Flattersatz
%\setlength{\RaggedRightParindent}{1em} % evtl. parskip


%% Sans Serif
%\usepackage{cmbright}
%\renewcommand{\familydefault}{\sfdefault}
%% Palatino
%\usepackage[sc]{mathpazo}
%\linespread{1.05}         % Palatino needs more leading (space between lines)
%\setkomafont{sectioning}{\normalcolor\bfseries} % Kapitel�berschriften

%%% Kapitel�berschriften: Mit gro�en Zahlen
%\usepackage{titlesec}
%\titleformat{\chapter}[display]
%{\bfseries\Large}
%{ %\Huge\textsc{\chaptertitlename} % f�r das Wort 'Kapitel'
%\hfill\fontsize{120}{70}\selectfont\color{lightgray}\textbf{\thechapter}}
%{-2ex}
%%{\filleft\fontsize{50}{70}\selectfont\scshape} % Kapit�lchen oder...
%{\filleft\fontsize{50}{70}\selectfont\textbf} % ...oder keine Kapit�lchen
%[\vspace{0ex}]
%
%%%% Part�berschriften
%\titleformat{\part}[display]
%{\bfseries\Large}
%{ %\Huge\textsc{\chaptertitlename} % f�r das Wort 'Kapitel'
%\hfill\fontsize{120}{70}\selectfont\color{lightgray}\textbf{\thepart}}
%{-2ex}
%{\filleft\fontsize{50}{70}\selectfont\scshape} % Kapit�lchen oder...
%%{\filleft\fontsize{50}{70}\selectfont\textbf} % ...oder keine Kapit�lchen
%[\vspace{0ex}]


\newcommand{\ER}{Erd\H{o}s-R\'enyi }
\newcommand{\BA}{Barab\'asi-Albert }
\newcommand{\mean}[1]{\left< #1 \right>}
\newcommand{\abs}[1]{\left| #1 \right|}
\newcommand{\norm}[1]{\lVert#1\rVert}
\newcommand{\mat}[1]{\mathbf{#1}}
\newcommand{\tgraph}{\mathcal{G}}

\theoremstyle{definition} % non-italic
\newtheorem{annahme}{Annahme} % braucht amsthm
\newtheorem{definition}{Definition}
\newtheorem{theorem}{Theorem}
\newtheorem{satz}{Satz}
\newtheorem{frage}{Frage}
%\input{watermarks/watermark.tex}
\DeclareMathOperator{\nnz}{nnz}

% + Graphicspath nach begin document

% aus Doi hyperref machen
%\newcommand*{\doi}[1]{\href{http://dx.doi.org/\detokenize{#1}}{doi: \detokenize{#1}}}




% The order of the parts in the document is only our suggestion,
% you can change it, if you wish.
% Don't put any other text between those commands.
% Do not remove the \*matter macros.
% Use standard macros to include new chapters.

%\definecolor[named]{Mokka}{rgb}{0.501,0.25,0}
%\addtokomafont{section}{\color{Mokka}}
%\addtokomafont{chapter}{\color{Mokka}}


%%%%%    \includeonly{chapters/Part1/01-Introtext}

\begin{document}
\graphicspath{{images_gnu_tex/}{/images}}

\selectlanguage{english}
    \frontmatter
        %\maketitle
        %\cleardoublepage

\null\vfill\itshape

\begin{flushright}
	Ich widme diese Arbeit \\
	meiner Familie und meinen Freunden
\end{flushright}
\thispagestyle{empty}
\upshape\cleardoublepage


        %\selectlanguage{english}
\begin{abstract}
Some Text.
\end{abstract}

\cleardoublepage


\selectlanguage{ngerman}
\begin{abstract}
Etwas Text.
\end{abstract}

\cleardoublepage
        \tableofcontents    
    \mainmatter
        % Part 1
        %\part{Setting the frame}

\chapter{Introduction}
%
\paragraph{Models for epidemics\color{Cayenne}{.}}
The understanding and prediction of the spread of infectious diseases are important issues in modern societies.
In human medicine, an important goal is the control of severe diseases such as AIDS.
On the other hand, a reduction of economic loss by harmless diseases, such as common cold or flu, is important for political economy.
In the field of agriculture, livestock epidemics are a major economic issue.
A prominent example is food and mouth disease, which caused tremendous economic losses in the UK in 2001 \citep{Kitching:2005dd}.

In the context of understanding the principles behind epidemic spreading processes, i.e. the way how a disease is transmitted through a population, \emph{conceptional} models are used.
They make use of simple assumptions for the local (person-to-person) dynamics and focus on the big picture of the process.
Conceptual models are very similar to models in theoretical physics, because they focus on the very essence of the problem (here: the macroscopic view, spreading patterns).
However, they have to neglect many details of the real problem (here: physiology, symptoms, individual behavior, infection pathways and many more!) in order to have mathematically feasible models.
Another important issue of epidemiology is the \emph{forecast} of epidemic spreading processes.
Forecast models incorporate as much information as possible and the main focus is not an understanding of the basic principles.

%\subsection{Development of mathematical epidemiology}
Major contributions to the modern theoretical framework were provided by \citep{kermack:27}, \citep{bailey:57} and \citep{andersonmay:92}.
In his review about the mathematics of infectious diseases Hethcote reports a model for smallpox was already formulated in 1760 by D. Bernoulli (see \citep{Hethcote:2000} and references therein).
In the early 20th century, people developed mathematical models for epidemics: a discrete time model in 1906 \citep{Hamer} and a differential equation model in 1911 \citep{Ross}.
The epidemic threshold (section \ref{sec:sir_model}) was found in the 1920s \citep{kermack:27}.
Starting from Bailey's book \citep{bailey:57} in the 1950s, the modeling of infectious diseases became a major scientific research field.
Modern models of infectious diseases include vaccination, demographic structure, disease vectors, quarantine and even game theory (\citep{Bauch:2004} and references in \citep{Hethcote:2000}).
The availability of contact data in recent years led to a strong impact on network analysis on epidemiology.
Well known concepts of mathematics (graph theory \citep{Bollobas:1985}) and social sciences (social network analysis \citep{WassermanFaust}) have been adopted to disease modeling, since the connections between individuals are related to their epidemic spreading potential \citep{Keeling:2005}.

\paragraph{Complex networks as spreading substrates\color{Cayenne}{.}}
%\section{Complex networks}
Data of gigantic scale have emerged by the proliferation of computerized data acquisition and storage volumes in recent years.
A significant portion of this data describes interactions between elements in a system.
Depending on the particular system, both the interactions and the elements themselves can have various meanings.
Thus, for example the structure of the world wide web can be described by links between web pages, or a biological system by the food transport between individual species.
The availability of data describing such systems has facilitated the emergence of modern network science.
Network science is concerned in the broadest sense with the development, description and development of complex networks, no matter what the network structure describes in particular.
Due to its applicability to many different systems, network science is an inherently interdisciplinary field.
Interdisciplinarity is not least important, because subject-specific meta-information is needed for the validation of a specific result.

Network science makes use of tools and methods known from \emph{graph theory}.
The elements of a network are called nodes and the interactions are links between the nodes.   
The mathematical roots of graph theory originally go back to L. Euler \citep{euler:1736}.
In 1735, Euler solved the so-called K�nigsberg bridge problem: the city of K�nigsberg was divided by a river into four quarters, which were connected by 7 bridges.
Is it now possible to cross each of the bridges exactly once in a closed walk?
Euler reduced quarters to nodes and bridges to edges of a network such that the edges connect the nodes.
He was able to show that there is no closed path that uses each bridge exactly once.
He also has the given the conditions that must be met for the existence of such a closed path in any network.
%%

Beyond the tools and methods of graph theory, the origin of modern network science goes back to psychology.
More specifically, the analysis of \emph{social networks} raised a lot of questions about the roles of particular individuals in these systems.
In fact, many of the measures of network theory have been defined in the psychological literature decades ago.



A review on networks is found in \citet{Newman2003}.
Analyses of livestock trade networks are in \citet{Christley:2005}
\citet{Bigras:2007}\citet{Green:2006}.

The interplay between aggregation window and spreading potential was analyzed in \citet{Bajardi:2012}.

In principle, networks can be recognized in many systems including a matrix model, i.e. any kind of interaction between elements.
However, the focus of this work and of other works reported here lies in the topological complexity of the systems.
Therefore, the term \emph{complex network} is frequently used in the literature in order to make a distinction between systems with sophisticated topological features and systems showing a simple topology, such as lattices.


Network science is capable of analyzing large scale systems.
Prominent examples are the internet or the world wide web.
Figure~\ref{fig:opte} shows a visualization of the internet on the physical level.
Note that the world wide web names the links between web pages, while the internet is the network of physical (or wireless) connections between routers.
\begin{figure}[htb]
\begin{center}
\includegraphics{opte.png}
\caption{Figure from opte.}
\label{fig:opte}
\end{center}
\end{figure}

This work has an explicit focus on \emph{paths} being a very essential property of networks.


        %\documentclass[openright,twoside,headsepline]{scrbook}
%\usepackage[applemac]{inputenc}
%\usepackage{graphicx,xcolor,hyperref} % obsolete in HU-diss
%\usepackage[round,authoryear]{natbib}
%\setlength\bibhang{2em} 
%
%
%\KOMAoptions{numbers=noenddot}
\usepackage{amsmath,amssymb,amsfonts,amsthm,epigraph,scrpage2}
\usepackage[ngerman,english]{babel}
\usepackage{centernot} % f�r Durchstreichung

\usepackage{currvita}
\renewcommand*{\cvheadingfont}{\large\bfseries} % CV Titel
\renewcommand*{\cvlistheadingfont}{\bfseries\sffamily} % sub-�berschriften
\renewcommand*{\cvlabelfont}{\sffamily} % items

\definecolor{Cayenne}{rgb}{0.502,0.0,0.0}
\definecolor{Steel}{rgb}{0.4,0.4,0.4}
\definecolor{Tri_blue}{rgb}{0.039,0.5098,0.8}
\definecolor{Tri_yellow}{rgb}{0.5529,0.5451,0.0549}


%\setcounter{secnumdepth}{3} % sub subsections numbering
%\setcounter{tocdepth}{3} % subsubsections inTOC

\usepackage[format=plain,singlelinecheck=false, font={sf,small},labelfont={bf,color=Steel}]{caption}
\DeclareCaptionLabelSeparator{cayenne_period}{\textcolor{Cayenne}{.} }
\captionsetup{labelsep=cayenne_period}

% Colors
\addtokomafont{chapter}{\color{Steel}}
\addtokomafont{section}{\color{Steel}}
\addtokomafont{subsection}{\color{Steel}}
\addtokomafont{subsubsection}{\color{Steel}}
\addtokomafont{paragraph}{\color{Steel}}

\addtokomafont{pagehead}{\color{Steel}}
\renewcommand{\pnumfont}{\color{Steel}} 
\addtokomafont{headsepline}{\color{Steel}} 
\pagestyle{scrheadings}

%  labels in description environments
\renewcommand{\descriptionlabel}{\hspace\labelsep{}\sffamily\small\bfseries{}\color{Steel}{}}

%\makeatletter % dot after sections and all below
%\let\std@sect\@sect
%\def\@sect#1#2#3#4#5#6[#7]#8{\std@sect{#1}{#2}{#3}{#4}{#5}{#6}[#7.]{#8\color{Cayenne}{.}}}
%\makeatother
\usepackage{acronym}

\usepackage[leftcaption]{sidecap} % inner, outer,left,right
\sidecaptionvpos{figure}{t}

% Papiergr��e
%\setlength{\paperwidth}{21cm}
%\setlength{\paperheight}{25cm}
%\recalctypearea
%\usepackage{geometry}
%\usepackage[cross,a4,center]{crop}

%% Flattersatz
%\usepackage[document]{ragged2e} % Flattersatz
%\setlength{\RaggedRightParindent}{1em} % evtl. parskip


%% Sans Serif
%\usepackage{cmbright}
%\renewcommand{\familydefault}{\sfdefault}
%% Palatino
%\usepackage[sc]{mathpazo}
%\linespread{1.05}         % Palatino needs more leading (space between lines)
%\setkomafont{sectioning}{\normalcolor\bfseries} % Kapitel�berschriften

%%% Kapitel�berschriften: Mit gro�en Zahlen
%\usepackage{titlesec}
%\titleformat{\chapter}[display]
%{\bfseries\Large}
%{ %\Huge\textsc{\chaptertitlename} % f�r das Wort 'Kapitel'
%\hfill\fontsize{120}{70}\selectfont\color{lightgray}\textbf{\thechapter}}
%{-2ex}
%%{\filleft\fontsize{50}{70}\selectfont\scshape} % Kapit�lchen oder...
%{\filleft\fontsize{50}{70}\selectfont\textbf} % ...oder keine Kapit�lchen
%[\vspace{0ex}]
%
%%%% Part�berschriften
%\titleformat{\part}[display]
%{\bfseries\Large}
%{ %\Huge\textsc{\chaptertitlename} % f�r das Wort 'Kapitel'
%\hfill\fontsize{120}{70}\selectfont\color{lightgray}\textbf{\thepart}}
%{-2ex}
%{\filleft\fontsize{50}{70}\selectfont\scshape} % Kapit�lchen oder...
%%{\filleft\fontsize{50}{70}\selectfont\textbf} % ...oder keine Kapit�lchen
%[\vspace{0ex}]


\newcommand{\ER}{Erd\H{o}s-R\'enyi }
\newcommand{\BA}{Barab\'asi-Albert }
\newcommand{\mean}[1]{\left< #1 \right>}
\newcommand{\abs}[1]{\left| #1 \right|}
\newcommand{\norm}[1]{\lVert#1\rVert}
\newcommand{\mat}[1]{\mathbf{#1}}
\newcommand{\tgraph}{\mathcal{G}}

\theoremstyle{definition} % non-italic
\newtheorem{annahme}{Annahme} % braucht amsthm
\newtheorem{definition}{Definition}
\newtheorem{theorem}{Theorem}
\newtheorem{satz}{Satz}
\newtheorem{frage}{Frage}
%\input{watermarks/watermark.tex}
\DeclareMathOperator{\nnz}{nnz}

% + Graphicspath nach begin document

% aus Doi hyperref machen
%\newcommand*{\doi}[1]{\href{http://dx.doi.org/\detokenize{#1}}{doi: \detokenize{#1}}}


%
%
%
%\begin{document}
%\graphicspath{{/Users/lentz/Documents/GitHub_locals/Thesis/images/}}
%\tableofcontents
%


\chapter{Theory}\label{sec:theory}
In this chapter, we review the mathematical formalism that is used to model infectious diseases and networks.
We define mathematical frameworks for the analysis of epidemics and networks in this chapter and summarize several relevant results of earlier research.
The modeling of infectious diseases makes extensive use of compartment models.
We address these models in sections \ref{sec:si_model} and \ref{sec:sir_model}.
Section~\ref{sec:network_theory} gives an overview over several results of modern network theory.
In addition, Appendix~\ref{sec:implementation} describes efficient computer implementations of networks.

\section{Models of infectious diseases}\label{sec:inf_diseases}
Before we formulate models for the spread of epidemic diseases, we have to differentiate between \emph{conceptional} models and \emph{realistic} disease models.
While the former class is used to provide conceptional results such as the computation of thresholds or testing theories \citep{Hethcote:2000}, realistic disease models use as many aspects as possible to provide a forecast of a particular spreading process.
Realistic disease models can be very complex and are beyond the scope of this work, hence we focus on the use of conceptional models.
In the following section we briefly report some properties of basic epidemic models following the lecture notes of \citet{Chasnov:2010}.

\subsection{SI model}\label{sec:si_model}
Let us consider a population of $N$ individuals.
In the simplest case, the infection status of each individual is either susceptible ($S$) or infected ($I$) and there are no births and deaths in the population.
Susceptible individuals become infected, if they are in contact with an infected%
\footnote{There is a distinction between infected and infectious, in general.
By definition, infected individuals are invaded by a pathogen and act as hosts for its multiplication.
On the contrary, only \emph{infectious} individuals have the ability to infect others and infected individuals are not necessarily infectious \citep{Rolle:2006vr}.
Nevertheless, we consider infected equivalent infectious throughout this thesis.}.%
In epidemiology, the classes susceptible and infected are called \emph{compartments} and every new infection increases the population of the infected compartment following the local reaction scheme
\[
S+I \rightarrow 2 I.
\]
This mimics the behavior of an infectious disease without immunization, i.e. infected individuals stay permanently infected.

Provided that $\alpha $ is the rate, under which new susceptible become infected, we obtain the corresponding differential equation model
\begin{align}\label{eq:si_model}
\frac{dS}{dt} &= -\alpha SI \nonumber \\
\frac{dI}{dt} &= \alpha SI,
\end{align}
where $S$ and $I$ are the numbers of susceptible and infected individuals respectively.
The model \eqref{eq:si_model} is called SI-model.
The total population is $N=S+I$.
Thus, \eqref{eq:si_model} can be rewritten as
\[
\frac{dI}{dt}=\alpha (N-I)I,
\]
i.e. a logistic differential equation.
Hence, in the limit $t\rightarrow \infty $ the whole population is infected ($I(\infty )=N$). 

\subsection{SIR model}\label{sec:sir_model}
In contrast to the infection dynamics introduced in the previous section, many epidemics include an immunized state, where immunized individuals do not contribute to disease spread.
Examples are measles or whooping cough \citep{andersonmay:92,grenfell:92}.
In these cases, individuals recover from the disease after being infected for a certain time period.
This behavior is modeled by the introduction of an additional compartment for the recovered population.
The infection scheme is extended to susceptible-infected-recovered (SIR) as in the following infection model \citep{kermack:27}:
\begin{align}\label{eq:sir_model}
\frac{dS}{dt} &= -\alpha SI \nonumber \\
\frac{dI}{dt} &= \alpha SI -\gamma I \nonumber \\
\frac{dR}{dt} &= \gamma I ,
\end{align}
where $\alpha $ is the infection rate and $\gamma $ is the immunization or recovery rate.
A typical solution of \eqref{eq:sir_model} is shown in Figure~\ref{fig:std_sir_model}.
There is no analytic solution for the system \eqref{eq:sir_model}, but some fundamental conclusions can be obtained analytically.

%
\begin{SCfigure}%[htbp]
%\begin{center}
\includegraphics{sir_model.pdf}
\caption{Solution of the susceptible-infected-recovered (SIR) model \eqref{eq:sir_model}.
The number of infected shows that the spreading process is a single event.
Note that a fraction of the population is still susceptible at the end of the process.
Parameters: $\alpha = 3$, $\gamma = 1$, $N=300$, $S_0=1$.}
\label{fig:std_sir_model}
%\end{center}
\end{SCfigure}
%

The SIR model shows more sophisticated features than the SI model \eqref{eq:si_model}.
To begin with, we analyze the fixed points of the system, i.e. $(S_*,I_*,R_*)$ where
\begin{equation}
\frac{dS_*}{dt} = -\alpha S_*I_* =0 ,\; \;\;
\frac{dI_*}{dt} = \alpha S_*I_* -\gamma I_* =0,\; \;\;
\frac{dR_*}{dt} = \gamma I_* = 0.
\end{equation}
It follows from the last equation that $I_*=0$ at the fixed point, where $S_*$ and $R_*$ can be arbitrary as long as $S_*+R_*=N$.
Hence, $(S_*,0,R_*)$ is a fixed point.

Let us first analyze the stability of the fixed point in the early phase of an infection.
Almost all individuals are susceptible and consequently $I_*=N-S_*$.
An outbreak occurs, if and only if $dI/dt >0$ in this phase, i.e.
\begin{equation}\label{eq:prelim_condition}
\frac{dI}{dt}=\alpha S_* (N-S_*) - \gamma (N-S_*)=(N-S_*)(\alpha S_* -\gamma ) >0.
\end{equation}
It follows from \eqref{eq:prelim_condition} that the number of infected grows, if
\begin{equation}\label{eq:prelim_rnod}
\alpha S_* / \gamma >1.
\end{equation}
Equation \eqref{eq:prelim_rnod} is extremely important in epidemiology, because it defines a threshold for the unfolding of an infection spreading process.
This fraction is called the \emph{basic reproduction number} $R_0$.
Recall that $S_* \approx N$ in the fixed point.
Thus, it follows that the outbreak condition is
\begin{equation} \label{eq:r0}
R_0 = N \frac{\alpha }{\gamma } >1.
\end{equation}

The basic reproduction number describes the average number of follow-up infections by each infected individual.
It is one of the main goals in epidemiology to bring down the basic reproduction number of a disease below the critical value $R_0=1$.
As one can immediately see from Equation~\eqref{eq:r0}, this can be done by reducing the infection rate $\alpha $ or by increasing the immunization rate $\gamma $.
This is the reason for the implementation of mass vaccination.
Vaccination basically decreases the size of the initial susceptible population $S_0=S_*$.
A reduction of the infection rate can be achieved by increasing hygiene standards or appropriate behavior, say wearing warm clothes in winter time to avoid common cold.
The immunization rate can be increased by drugs.

Let us now focus on the late phase of an SIR-infection.
In contrast to the SI-model of Section~\ref{sec:si_model} an SIR like outbreak does not necessarily infect the whole population, even if $R_0>1$.
The reason is that there has to be a critical mass of susceptible individuals in order to keep an infection alive (see Equation~\eqref{eq:prelim_rnod}).
The total number of infected during an infection given by the number of recovered at the end of the infection, since every recovered has to be in the infected state in the first place.
A central measure throughout this work is therefore the \emph{outbreak size} $R_\infty$.

To compute the outbreak size, we consider the second fixed point of \eqref{eq:sir_model}, i.e. the fixed point for $t \rightarrow \infty $.
At this point there are no infected and a fraction of the population is recovered.
Hence, the fixed point is $(N-R_\infty , 0, R_\infty )$.
A simple way to obtain the outbreak size $R_\infty $ is to use equations \eqref{eq:sir_model} and compute 
\[
\frac{dS}{dR}=-\frac{\alpha }{\gamma } S 
\]
and separate the variables \citep{Chasnov:2010}.
This yields
\[
\int _{S_*} ^{N-R_\infty} \frac{dS}{S}=-\frac{\alpha }{\gamma } \int _{R_*} ^{R_\infty} dR .
\]
We integrate from the initial condition at $t=0$ to the final condition at $t \rightarrow \infty$, where $S_\infty = N-R_\infty $.
Using that $R_* =0$ at $t=0$ gives 
\begin{equation}\label{eq:transcendental}
R_\infty = S_*-S_* e ^{-\frac{\alpha}{\gamma}R_\infty}.
\end{equation}
This transcendental equation can be solved numerically using a Newton-Raphson technique.
The outbreak size $R_\infty $ only takes finite values for $\alpha / \gamma > 1$.
A solution of Equation~\eqref{eq:transcendental} is shown in Figure~\ref{fig:transcendental}
%
\begin{figure}[htbp]
\begin{center}
\includegraphics{R_0_R_infty.pdf}
\caption{Relative outbreak size vs. basic reproduction number.
The outbreak size takes finite values only for $R_0/N >1$.
Note that even for supercritical $R_0$ the outbreak size is in general smaller than the total population.
}
\label{fig:transcendental}
\end{center}
\end{figure}

It should be noted that an SIR epidemic is a single event, i.e. it possesses a \emph{characteristic time scale}.
The analysis of the late phase of an epidemic also gives information about these time scales.
Let us consider the second equation of \eqref{eq:sir_model}.
\begin{equation}\label{eq:sir_model_only_i}
\frac{dI}{dt} = \alpha SI -\gamma I
\end{equation}
In the late phase of an SIR-type epidemic, the fraction of infected is small.
Given sufficiently large values of $R_0$, the fraction of susceptible is also small in this phase (see Figure~\ref{fig:transcendental}).
Thus, we neglect the quadratic term in \eqref{eq:sir_model_only_i}.
This gives $\frac{dI}{dt} = -\gamma I $, which has the solution
\begin{equation}\label{eq:sir_characteristic_time_scale_gamma}
I(t)=I(0)e^{-\gamma t}.
\end{equation}
Hence, the infection decays exponentially for large $t$ and the inverse recovery rate $1/\gamma $ defines the characteristic time of the epidemic.

A similar concept to the SIR model is the SIS model, where infected individuals return to the susceptible state after a certain period.
Being a single-event model, the SIS model has many similarities to the SIR model.
The most crucial difference is that SIS models show an endemic state for $t\rightarrow \infty $, i.e. both $S$ and $I$ take finite values in the long term so that fraction of infected remains in the system permanently.

\subsection{Force of infection}\label{sec:force_of_infection}
The model presented in Section~\ref{sec:sir_model} describes only the very basic behavior of epidemic dynamics, and is therefore a conceptional model.
However, it is one of the main objectives in epidemiology to have an understanding of the explicit \emph{infection rates} in the process.
Depending on their detailed structure, the infection rates themselves can cause complex infection dynamics.

The term $\alpha I$ used in $\alpha S I$ in the second equation of \eqref{eq:sir_model} is a special, very simple case of an infection rate.
It corresponds to the case where every susceptible is in contact with every infected in the population.
More generally, we have to replace $\alpha I$ by an abstract infection rate $\lambda $ containing more information about the interaction between susceptible and infected individuals \citep{Keeling:2005}.
Thus, the equation for the infected becomes
\[
\frac{dI}{dt} = -\lambda S -\gamma I.
\]
The rate $\lambda $ is called the \emph{force of infection}.
In principle, this parameter can be arbitrarily complex, because it contains detailed information about the mixing properties of the population.
This information can be represented as contact networks, demographic contact structures, etc.

In most cases, detailed information about mixing is not available.
Instead, we assume \emph{random mixing} of the population, i.e. every individual can be in contact with every other individual.

Considering a contact rate, where each individual has a small chance of being connected to any other individual in the population yields a transmission rate \citep{Keeling:2005}
\begin{equation}\label{eq:force_of_infection}
\lambda = \tau n \frac{I}{N}\equiv \beta \frac{I}{N},
\end{equation}
where $\tau $ is the transmission rate, $n$ is the effective contact rate and $I/N$ is the fraction of infectious contacts.
The factor $1/N$ can be interpreted as the ``contact surface'' between the susceptible and infected population.
It is reasonable to replace the infection term $\alpha $ in \eqref{eq:sir_model} by $\beta /N$ to explicitly include the force of infection.
The results presented in Section~\ref{sec:sir_model} remain qualitatively the same.

Although the force of infection gives a more reasonable description of the infection process, the assumption of random mixing remains inappropriate for many real world systems.
Due to the availability of contact data, the random mixing assumption can be improved in terms of contact networks.
Even if the exact data of an epidemic system is not available, research on complex networks allows us to give more realistic models about mixing.
In the next section, we briefly report important results in complex network research and focus on the interplay between networks and epidemics in Section~\ref{sec:epi_networks}.

\section{Network theory}\label{sec:network_theory}
As we have pointed out in the previous section, standard epidemic models make use of the random mixing assumption.
This assumption holds, if no further information about the contact structure within a population is available.
The random mixing assumption yields a worst case scenario of the infection dynamics.
Even an overestimation of the outbreak size can be corrected by introducing smaller, effective disease parameters.
However, the random mixing assumption does not allow for non homogeneous mixing, since each individual is considered equal.
The equality of links between individuals is not a reasonable assumption for many epidemic substrates.
Examples of epidemic substrates are contact structures of humans, livestock trade or links between computers.
Apparently, connections are not purely random in these systems so that there are certain rules for the occurrence of links.

%\paragraph{Main observations\color{Cayenne}{.}}
The main contribution of network science to epidemiology is that it allows for the analysis of detailed contact structures.
If detailed information about the contact structure is available, the random mixing assumption is obsolete.
Instead, the system can be treated using the underlying contact structure in form of a network.
Since the beginning of the 21st century, large amounts of data about these contact structures have become available for social, economic, transportation, and biological networks.
Observations showed that many real-world networks share common topological properties, which are described in Section~\ref{sec:macro_measures}).
Since the number of their non-trivial topological properties is considerable, they are often referred to as \emph{complex} networks.

%\paragraph{Research field: network science\color{Cayenne}{.}}
Modern network science is an interdisciplinary research field, because it addresses systems of diverse scientific affinity.
Its roots lie in graph theory (mathematics) and social network analysis (social sciences).
Social network analysis plays a particular role for the definition of local network measures (see Section~\ref{sec:micro_measures}), whereas the influence of graph theory is stronger in macroscopic problems as percolation or statistical properties in the thermodynamic limit.
An important focus of network science is to find common features of different networks and to explore the basic principles behind their emergence.
Applied network science makes extensive use of methods used in computer science.
A brief introduction to efficient computer methods for network analysis is provided in Appendix~\ref{sec:implementation}.

\subsection{Matrix representations}\label{sec:network_matrices}
A network is a system of nodes that are connected by edges.
Edges can be undirected, directed and weighted.
In principle, a network can consist of edges of different types.
In this case, the network can be represented by multiple networks sharing the same set of nodes, but different edges.

Networks are called graphs in mathematical literature.
A graph $G=(V,E)$ is a set of nodes (or vertices) $V$ and edges (or arcs) $E$, where each edge is given by the tuple of nodes it connects, i.e. $e_1 =(u,v) \in E$ connects nodes $u$ and $v$.
An edge $(u,v)$ being present in an undirected network implies the existence an edge $(v,u)$.
Apparently, this does not hold in directed networks.
In weighted networks, the edges carry additional information -- such as their importance, capacity, number of transported items or the geographical distance between the nodes they connect.
%
\begin{SCfigure}%[h]
%\begin{center}
\includegraphics{Simple_DiGraph}
\caption[Adjacency matrix.]{A simple directed network. The corresponding adjacency matrix is
\[
\mathbf{A}=\left(\begin{array}{cccc}0 & 1 & 1 & 0 \\0 & 0 & 0 & 1 \\0 & 1 & 0 & 0 \\0 & 1 & 1 & 0\end{array}\right).
\]
}
\label{fig:simple_digraph}
%\end{center}
\end{SCfigure}

Graphs can be represented by different graph matrices, where each matrix representation emphasizes typical properties of the network.
The most common graph matrix is the \emph{adjacency matrix} $\mathbf{A}$ with entries
\begin{equation}\label{eq:adjacency_matrix}
a_{ij}\equiv (\mathbf{A})_{ij}= 
\begin{cases}
1 & \text{if $i$ is connected to $j$} \\
0 & \text{else,}
\end{cases} 
\end{equation}
where the indices correspond to node labels.
An adjacency matrix contains the edges of the graph and can be seen of the most fundamental graph representation.
Figure \ref{fig:simple_digraph} shows a simple example of a directed graph and its adjacency matrix.
The corresponding matrix would be symmetric in the undirected case.
Weighted networks can be represented by weight matrices, where the values of the entries in \eqref{eq:adjacency_matrix} are not restricted to 0 and 1.

The adjacency matrix of an undirected network is always symmetric, because every non-zero entry $a_{ij}=1$ implies an edge into the opposite direction, $a_{ji}=1$.
Entries on the main diagonal $a_{ii}$ correspond to nodes with self loops, i.e. nodes with edges pointing back to themselves.
The $i$-th row the adjacency matrix contains non-zero entries $a_{ij}=1$, wherever node $i$ is connected to node $j$.
Hence, every row can be interpreted as the neighborhood of one node.
This holds for undirected and for directed networks.
The columns of $\mathbf{A}$ give the same information as the rows in the undirected case.
In directed networks, however, rows contain the out-neighborhood of each node and columns contain the in-neighborhood, respectively.

Information about paths of a certain length can be obtained using the powers of the adjacency matrix.
The adjacency matrix contains information about the number of paths of length $1$ between node pairs.
Evidently, the number of paths of length $2$ between two nodes $i$ and $j$ is given by $(\mathbf{A}^2)_{ij}$.
This applies also to paths of arbitrary length $n$ using the elements of $\mathbf{A}^n$.

An important example for weighted network matrices is a \emph{Markov chain}.
A Markov chain is a random process without memory and with a discrete state space and discrete time.
It is called time-homogeneous, if the transition rates are constant.
Time-homogeneous Markov chains can be represented as weighted networks and the corresponding weighted adjacency matrix is the \emph{transition matrix}.
Transition matrices are stochastic matrices, i.e. the elements of every row sum up to unity.
Each node represents a different state of the system and each edge is weighted with the probability to transition into the other state adjacent to the edge.
It is obvious that a transition matrix representation is useful to describe random walks on networks.
An example of such a process is shown in Figure~\ref{fig:markov_chain}.
The figure shows a drunkard toddling randomly in the left or right direction.
The underlying network represents a line of locations, where the drunkard can be located.
At every time-step there is a certain probability to move to another location.
%
\begin{SCfigure}%[htbp]
%\begin{center}
\includegraphics{Markov_chain.pdf}
\caption{Trajectory of a toddling drunk man as an example of a Markov chain.
At every location there is a probability for the drunkard to go left or right.
The node rightmost node is an absorbing state and could model a park bench.
Weights at arrowheads mark the transition probability.
(inspired by \citep{Aldous_book}).}
\label{fig:markov_chain}
%\end{center}
\end{SCfigure}
%
The state of the random walker can be described by a probability vector $\mathbf{p}$, where the initial state of Figure~\ref{fig:markov_chain} is $\mathbf{p}=(0,1,0,0)$.
The transition matrix $\mathbf{M}$ is a weighted adjacency matrix as it follows from the figure.
Given a state $\mathbf{p}_{t}$ at time $t$, the state of the next time step is given by $\mathbf{p}_{t+1}=\mathbf{p}_t \mathbf{M}^T$.
The equilibrium state $\mathbf{p}_\mathrm{eq}$ follows in the limit $\lim _{t\rightarrow \infty } \mathbf{p}_0 (\mathbf{M}^T)^t$, i.e. the equilibrium state is given by the dominant eigenvector of $\mathbf{M}$.

As a special case of transition matrices, the author would like to mention the \emph{Google matrix}.
It describes a random walk on a network, but allows for shortcuts to any node in the network with a certain probability.
The eigenvectors of Google matrices are used for the computation of node rankings according to the PageRank-Algorithm \citep{PageRank:}.

Finally, the \emph{Laplace-matrix} of a network is an appropriate representation to model diffusion processes on networks.
For undirected networks the Laplace-matrix is defined as
\begin{equation}\label{eq:laplace_matrix}
\mathcal{L}=\mathbf{D}-\mathbf{A},
\end{equation}
where $\mathbf{A}$ is the adjacency matrix and $\mathbf{D}$ is a diagonal matrix containing the degree $d_i=\sum _j a_{ij}$ of each node.
The definition \eqref{eq:laplace_matrix} has strong analogies to the discrete Laplace-Operator \citep{Press:1992}.
Consequently, it can be used to model diffusion processes on graphs in analogy to Laplace operators in continuous systems (see Section~\ref{sec:PRE}).
The spectra of adjacency and Laplace matrices also contain information about the evolution/history of networks \citep{Banerjee2009}.

\subsection{Network measures}\label{sec:network_measures}
Before we address ourselves to models of real world networks, we may introduce methods to measure structural properties of networks.
On the microscopic scale, this can be done in terms of \emph{node centrality} measures.
These measures are crucial to assess the importance of single nodes in the network.
On the macroscopic scale, we are interested in the large-scale properties of networks, i.e. percolation, distributions of centralities, connected components, or other large scale structures.

Implementations of appropriate data structures for the computation of network measures are briefly summarized in Appendix \ref{sec:implementation}.

\subsubsection{Network terminology}\label{sec:network_terminology}
Let $G=(V,E)$ be a graph consisting of a set of nodes $V$ and a set of edges $E$.
We denote the number of nodes in the network by $N=\abs{V}$ and the number of edges by $m=\abs{E}$.
Every route across a graph along its edges without repeating nodes is called a \emph{path}.
Each path is given by an ordered set of the nodes traversed, i.e. $(v_1,v_2,\dots ,v_l)$, with $v_i \in V$ and all traversed edges are in $E$, i.e. $v_i,v_{i+1} \subseteq E$ for all $i$.
A \emph{shortest path} between a node pair is given by the smallest set of nodes connecting it.
In general, there exist multiple shortest paths between nodes.
If there is a path from every node in the network to any other node, the network is called \emph{connected}.
In directed networks, we have to consider two types of connectedness.
A directed network is strongly connected, if there is a directed path between all node pairs and weakly connected, if the node pairs would be connected ignoring the direction of edges.

The \emph{distance} between two nodes is the length of the shortest path between them and the longest distance between all node pairs is the \emph{diameter} $D$ of the network.
Every closed path is called a \emph{cycle}.
Graphs that do not contain cycles are called acyclic graphs or \emph{trees}.
The neighborhood of a node $u$ is the set of all nodes adjacent to it and the size of the neighborhood is the \emph{degree} of the node.
Hence, a node $v$ is in the neighborhood of $u$, if $(u,v) \in E$.
We distinguish between in-degree and out-degree in directed networks.
Finally, $G_0=(V_0,E_0)$ is a \emph{subgraph} of $G=(V,E)$, if $V_0 \subseteq V$ and $E_0 \subseteq E$.

\subsubsection{Microscopic measures}\label{sec:micro_measures}
Given a network, an important question is, if some nodes are more important than others.
Therefore, we summarize several measures of node \emph{centrality}.
The idea of centrality mainly goes back to social network analysis \citep{Granovetter:1973wj,Freeman,WassermanFaust}, but has been widely adopted and extended in network science.
We restrict ourselves to those measures, that are indispensable when describing networks.
A more exhaustive overview of centrality measures is found in the review article \citep{MartinezLopez2009} or in online documentation of network analysis software, e.g.~\citep{hagberg2008,networkx:}.
In the following, $N$ denotes the order of the network (the number of nodes) and $m$ the number of edges.
%
\begin{figure}[htb]
\begin{center}
\includegraphics{DiGraph-Centrality}
\caption{A directed network for the demonstration of different centrality measures.}
\label{fig:example_net}
\end{center}
\end{figure}
%

\paragraph{Degree\color{Cayenne}{.}}
The simplest centrality measure is the degree $k$ of a node, which is the number of its neighbors.
In directed network, we distinguish between in-degree $k^-$ and out-degree $k^+$.
The degree follows immediately from the adjacency matrix, i.e.
\[
k ^-(i) = \sum _j a_{ji} \quad \text{and} \quad k ^+ (i)= \sum _j a_{ij}
\]
is the in- and out-degree of node $i$, respectively.
As an example, node $8$ in Figure~\ref{fig:example_net} has $k ^+(8)=4$ and $k ^- (8)=1$.
In weighted networks, the degree is computed in the same manner using a weight-matrix and is called in-weight and out-weight, respectively.

The degree centrality (sometimes normalized by its maximum value $N-1$) is used in a huge variety of applications.
One of its most important applications is to measure the heterogeneity of network connections, i.e. the existence of hubs in the network.
Hubs are nodes with a degree much larger than the rest of the system.
The heterogeneity of networks can be measured in terms of degree distributions.
We discuss the role degree distributions in Sections~\ref{sec:macro_measures} and \ref{sec:BA_model}.

\paragraph{Closeness\color{Cayenne}{.}} 
The closeness of a node $i$ is the reciprocal average distance to all other nodes that can be reached from $i$.
It can be normalized, so that the closeness is $1$, if all other nodes are reachable within one step and $0$ in the limit of infinite distances to all other nodes.
The closeness of a node $i$ in a network of order $N$ is defined as follows:
\begin{equation}\label{eq:closeness}
c(i)=\frac{N-1}{ \sum _j  d_{ij}}
\end{equation}
where $d_{ij}$ is the distance between nodes $i$ and $j$.
Some tools for an efficient computation of shortest-path distances are summarized in Section~\ref{sec:implementation}.
It should be noted that the distance between two nodes is defined to be infinite, if they are located in different components.
In this case, the corresponding terms are ignored and do not contribute to the sum in Equation~\eqref{eq:closeness}.
Thus the closeness is computed for each connected component separately.

Closeness centrality is capable of identifying nodes with short average path lengths to other nodes in the network.
Identifying high-closeness nodes is therefore reasonable for network navigation.
This holds in particular, if the exact route to the destination is unknown, because nodes with high closeness are probable to reach many destinations quickly.
In \citep{Zweig:closeness} it was shown that nodes of high closeness can act as efficient landmarks for navigation.

\paragraph{Betweenness\color{Cayenne}{.}}
In order to identify nodes that act as bridges between two subgraphs, the measure of betweenness was developed.
In Figure~\ref{fig:example_net}, node $4$ plays such a role.
It is characteristic for these nodes to contain a relatively large number of shortest paths that have to cross them.
Therefore, betweenness of a node $i$ is defined as
\begin{equation}\label{eq:betweenness}
b(i)=\sum _{s\neq i \neq t} \frac{\sigma _{st}(i)}{\sigma _{st}}
\end{equation}
where $\sigma _{st}$ is the number of shortest paths between nodes $s$ and $t$ and $\sigma _{st} (i)$ is the number of shortest paths between $s$ and $t$ going through node $i$.
The computation of betweenness is expensive using Equation~\eqref{eq:betweenness} directly.
Therefore, an efficient algorithm was introduced by Brandes \citep{Brandes:2001p2757}.

Note that bridge nodes might look ordinary in the first place, e.g. they could have only a few links.
However, removing node $5$ in Figure~\ref{fig:example_net}, for instance, would divide the network into two disjoint subgraphs with nodes $V_1=(1,2,3)$ and $V_2=(5,6,7,8,9)$ respectively.
Therefore, removing nodes of high betweenness from the network has been proven useful in order to divide networks into smaller components \citep{girvan2002,Newman:2004}.

\paragraph{Eigenvector centrality\color{Cayenne}{.}}
The idea of eigenvector centrality can be easily captured recalling the Markov chains described in Section~\ref{sec:network_matrices}.
Frequent iterative multiplication of the transition matrix $\mathbf{M}$ with a random vector gives the largest eigenvector of $\mathbf{M}$.
This relation is known as power method or von Mises iteration \citep{van_mises}.
The dominant eigenvector of the transition matrix gives the equilibrium state of the system.
Using this state as a measure of centrality assigns every node with the probability to find a random walker there after a long period.
The principle behind the dominant eigenvector of an adjacency matrix $\mathbf{A}$ is that important nodes are likely to be connected to other important nodes.
This recursive concept is reflected in the equation
\[
x_i =\frac{1}{\lambda } \sum _j a_{ij} x_j ,
\]
where $x_i$ is the centrality of $i$, $\sum _j a_{ij} x_j$ is the centrality of the neighborhood of $i$ and $\lambda $ is a constant.
This equation can be written as
\begin{equation}
\mathbf{Ax}=\lambda \mathbf{x}.
\end{equation}
It follows from the Perron-Frobenius-Theorem that $\lambda $ must be the largest eigenvalue of $\mathbf{A}$ in order to guarantee all entries of $\mathbf{x}$ to be positive \citep{Bonacich:1972,Bonacich:2007}.
The theorem guaranties unique solutions only for adjacency matrices of connected networks.
Hence, eigenvector centrality is only defined for connected graphs.
Nevertheless, the eigenvector centrality can be computed for each component separately, if a graph is not connected \citep{Bonacich:2007}.
Two widely used variants of eigenvector centrality allowing for disconnected networks are the PageRank and HITS algorithm \citep{Kleinberg:1999,PageRank:}.

\paragraph{Node components and range\color{Cayenne}{.}}
The component of a node is the set of nodes it is connected to by a path of any length.
We call the size of this set the \emph{range} of a node \citep{Lentz:2012pre}.
In directed networks, we distinguish between the out-component and in-component of a node.
The size of the former is its range and the size of the latter is its reachability.
Reachability measures the vulnerability of nodes against disease outbreaks in the network.
Given a network $G=(V,E)$ of $N$ nodes, the range of a node $v \in V$ is defined as
\begin{equation}\label{eq:range_def}
\mathrm{range}(v)=\frac{\left| H \right| }{N}, \; \text{ where } \; H=\{u \in V: v\rightarrow u \},
\end{equation}
where $ v\rightarrow u $ means that there exists a path from $v$ to $u$.
The reachability of a node is its range in the inverse graph $G^{-1}=(V,E^{-1})$, in which the directions of all edges are reversed. 

Apparently, the range of a node is of major importance for any epidemiological problem on a network, because it defines an upper bound for the size of any outbreak starting at this very node. 
Although the range measure is rather simple, it can show an interesting distribution.
The shape of its distribution is inherently related to percolation properties of the network.
We discuss this relation in Section~\ref{sec:network_analysis}.

\subsubsection{Macroscopic measures}\label{sec:macro_measures}
In order to obtain a macroscopic view of a network, we discuss measures that capture its large scale properties.
The central question for the analysis of real-world networks is, whether different networks share similar large-scale features or whether each network is unique.
In principle, the distribution of any centrality measure could yield insights into the macroscopic network structure.
As a matter of fact, the degree distribution of a network has been proven useful for the classification into different network types.
Therefore, we restrict ourselves to a discussion of the degree distribution being the most representative centrality distribution.

\paragraph{Degree Distribution\color{Cayenne}{.}}
In the simplest case, that all nodes of a graph have the same degree, the graph is called \emph{regular}.
These objects are also called regular lattices.
In this case, the degree distribution collapses to a single peak without statistical variation.

Observations of real-world networks have shown that some networks exhibit exponential decaying degree distributions, i.e. there is a variance of degrees, but the system possesses a \emph{typical degree}.
Examples are social networks and technological and economic networks, such as electric power-grids and traffic networks \citep{Amaral:2000,indian_railway}.

The nodes of the vast majority of large real-world networks, however, show a degree variation over several orders of magnitude.
Examples are networks of internet routers \citep{Faloutsos:1999}, links in the world-wide-web \citep{Barabasi99}, or scientific citations \citep{Price:1965}.
Their degree distributions are approximated by \emph{power-laws} of the form
\begin{equation}\label{eq:scale-free_distr}
P(k) \propto k^{-\gamma } ,
\end{equation}
where $2<\gamma <3$ for most observed networks \citep{all_scale_free_are_sparse,Newman2003}.
The approximation is reasonable for the tails of the distributions, i.e. for large values of $k$.
The identification of power-law distributions in empirical data is discussed in \citep{Clauset:2009}.

Distributions of the form \eqref{eq:scale-free_distr} are called \emph{scale-free}, because they do not allow for a meaningful detection of a typical value.
Instead, the network has a number of nodes with only a few neighbors and at the same time hubs with very large degrees.
The structural difference between random and scale-free networks is sketched in Figure~\ref{fig:random_scalefree}.
%
\begin{figure}[htbp]
\begin{center}
\includegraphics{exponential_scalefree}
\caption{Structural difference between networks with exponential (left) and scale-free degree distribution (right).
All nodes have a similar degree in the network with exponential desire distribution, while the scale-free network shows hubs with a significantly larger degree than the average.
Hubs are highlighted in red.}
\label{fig:random_scalefree}
\end{center}
\end{figure}

Scale-free networks have attained remarkable attention in the last years and many real-world networks have been conjectured as scale-free \citep{Barabasi99,Newman2003}.
Important consequences of this classification were found to be a change in the threshold behavior of epidemic processes \citep{Pastor-Satorras_vespi:2001} and their topological resilience to node failures \citep{Albert:2000}.
The degree distributions of collaboration networks and others were well fitted by a scale-free distribution with a sharp cut-off \citep{Newman:2001p,RevModPhys.74}, where the distribution takes the form $P(k)\propto k^{-\gamma }e^{-k/\kappa }$ with fitting constants $\gamma $ and $\kappa $.
\citeauthor{Amaral:2000} suggest the aging of nodes as a possible explanation for the existence of an exponential cut-off, indicating that real systems possess a natural upper bound for their number of links \citep{Amaral:2000}.

\paragraph{Clustering coefficient\color{Cayenne}{.}}
The idea of the clustering coefficient comes from social networks and was first mentioned in \citep{Milgram:1967}.
It measures, whether a network contains a significantly large number of triangles.
This behavior is conjectured to be typical for social networks and has the simple meaning:
``a friend of your friend is likely to be your friend''.
The clustering coefficient $C$ is the number of connected triples ($A- B - C - A$) divided by the actual number of triples ($A- B - C $) in the network.
Using the adjacency matrix $\mathbf{A}$, the clustering coefficient can be computed as follows:
\begin{equation}\label{eq:clustering_coefficient}
C=\frac{\operatorname{tr}(\mathbf{A}^3)}{\operatorname{sum} (A^2) -\operatorname{tr}(\mathbf{A}^2)},
\end{equation}
where $\operatorname{tr}(\mathbf{A} )$ denotes the trace of $\mathbf{A}$ and $\operatorname{sum} (\mathbf{A})=\sum _{ij} a_{ij}$ is the sum over all elements of $\mathbf{A}$.
In this work, we focus on the clustering coefficient as a macroscopic property of networks.
It should be noted that there is also a \emph{local} clustering coefficient defined by $c_i=\sum _{jl} a_{ij}a_{jl}a_{li}/(k_i (k_i -1))$ \citep{Watts:1998,dynamical_processes}.
Thus, a network clustering coefficient can also be defined by averaging over all local clustering coefficients $\mean{c_i}$, which gives slightly different values than \eqref{eq:clustering_coefficient} and should not be mixed up with the latter.

The clustering coefficient plays an essential role in the small-world model of networks \citep{Watts:1998}.
We discuss this model in Section~\ref{sec:network_models}.
In addition, significant clustering coefficients have been measured in social networks \citep{Holland1971}, but also in many other real-world networks \citep{Newman2003}.

\paragraph{Average shortest path length\color{Cayenne}{.}}
The elements of the distance matrix $d_{ij}$ represent the distance between nodes $i$ and $j$ in the network.
Ignoring those node pairs with infinite distance (i.e. setting $d_{ij}=0$) gives the average shortest path length
\begin{equation}\label{eq:}
l=\frac{1}{N(N-1)}\sum _{i, j} d_{ij}
\end{equation}

It is a common feature of many networks that the average shortest path length is much smaller than the number of nodes in the network, i.e. typically networks contain shortcuts \citep{RevModPhys.74}.
An early and impressive example was shown by \citeauthor{Milgram:1967}, where the average distance between two randomly chosen people in the united states was measured to be 6 \citep{Milgram:1967}. 
This property is called \emph{small world} phenomenon.
It is an important building block of the Watts-Strogatz network model, which we discuss in Section~\ref{sec:watts_strogatz_model}.

\paragraph{Connected components\color{Cayenne}{.}}
A connected component $G_\mathrm{cc}=(V_\mathrm{cc},E_\mathrm{cc})$ is a subgraph of $G=(V,E)$, where there is a path between any node pair in $V_\mathrm{cc}$.
In directed graphs, a connected component in the sense above is called \emph{strongly connected}.
A component is called \emph{weakly connected}, if it is connected ignoring the direction of edges.
Many real-world networks contain a dominant \emph{largest connected component} (LCC) that is typically much larger than all other components of the system.
This component is therefore also called \emph{giant component}.

In fact, the emergence of a giant component in a network is a second-order phase transition and is a graph theoretical percolation process \citep{Newman2003}. 
Components play an important role for epidemic processes, because the component membership of each node defines the maximum outbreak size of any epidemic started at this very node.
The general component structure of directed networks is discussed in \citep{Dorogovtsev:2001jd} and we provide further discussion of their epidemiological relevance in Section~\ref{sec:components_ranges}.

\paragraph{Accessibility\color{Cayenne}{.}}
If we directly connect each node of a network with all other nodes it is connected to by any path, we obtain the \emph{accessibility} of the network.
Accessibility measures the ability to reach destinations, which is of particular importance for transportation systems \citep{Garrison:1960up,mackewitz}.
Mathematically, we define the accessibility graph (also \emph{transitive closure}) of a network as follows:
Let $G=(V,E)$ be a network.
Than $G^*=(V,E^*)$ is the accessibility graph of $G$ with $(u,v) \in E^*$, if there is a path from $u$ to $v$.
The accessibility graph is typically dense, because it contains many more edges than the underlying network.
A (weighted) adjacency matrix $\mathbf{C}$ of $G^*$ for a $N$-node network is given by the cumulative matrix
\begin{equation}\label{eq:cumulative_matrix}
\mathbf{C}= \sum _{i=1} ^{N-1} \mathbf{A}^i,
\end{equation}
where $\mathbf{A}$ is the adjacency matrix of $G$ and the elements of $\mathbf{C}$ contain the actual number of paths between each node pair.
Consequently, we obtain the adjacency matrix $\tilde{\mathbf{C}}$ of the accessibility graph, when we normalize the elements $c_{ij}$ of the matrix defined in \eqref{eq:cumulative_matrix}, i.e.
\begin{equation}\label{eq:normalize_cumu_matrix}
\tilde{c}_{ij}=%
\begin{cases} 
1 & \text{ if }\; c_{ij}\neq 0 \\
0 &  \text{ if } \; c_{ij}=0 .
\end{cases}
\end{equation}

\section{Network models and epidemiology}\label{sec:network_models}
The analysis of real-world networks in terms of the measures introduced in Section~\ref{sec:network_theory} has given useful insight into the structural properties of these systems.
In particular, observations showed that many networks have heavy-tailed degree distributions and show non-vanishing clustering coefficients.
In this section we review the results of some widely used network models.
Neglecting higher order link correlations, most network models in this section are entirely defined by their degree distributions.
They are therefore generic realizations of ensembles with fixed $P(k)$.
At the end of the section, we give a comparison between the different models and discuss their relevance in epidemiology.

\subsection{Lattice model}
Lattice models are inherently related to homogeneously distributed geographical positions of individuals.
They show a high degree of regularity and their potential for SIS and SIR spreading processes has been studied in \citep{Harris:1974} and \citep{Bak:1990}, respectively.
The impact of the heterogeneous susceptibilities has been studied in \citep{Sander2002293}.
It was found that this heterogeneity introduces a broadening of the critical region and the outbreak threshold can be increased in the case of heterogeneous susceptibilities.


\subsection{Erd\H{o}s-R\'enyi model}\label{sec:er_model}
The Erd\H{o}s-R\'enyi model makes use of probabilistic methods to analyze network properties and is therefore a random graph model.
A \emph{random network} is generated by creating a set of $N$ nodes and connecting each of the $\frac{1}{2}N(N-1)$ possible node pairs\footnote{We focus on undirected networks here. In the directed case, there are $N(N-1)$ possible node pairs.} with a certain probability $p$.
Networks generated this way are often called $G_{N,p}$ networks, although they are in the proper sense elements of a $G_{N,p}$ \emph{ensemble}\footnote{A similar approach is to consider a fixed number of edges $m$ instead, yielding a $G_{N,m}$ ensemble.}.

Random graph theory addresses questions about typical properties of networks with an infinite number of nodes, i.e. networks in the thermodynamic limit $N\rightarrow \infty $.
Consequently, the edge occupation probability $p$ is the key parameter in random graph theory.
Properties of particular interest are the average shortest path length or the distributions of degrees, component sizes (percolation) and the occurrence of special subgraphs such as triangles.
Apparently, the expected number of edges in the network is $\mean{E}=\frac{1}{2} \, pN(N-1)$, if $p$ is the edge occupation probability.
In addition, every edge increases the degree of two nodes, so that the \emph{average degree} of a random network of $N$ nodes is
\begin{equation}\label{eq:mean_er_degree}
\mean{k}=\frac{2\mean{E}}{N}=(N-1)p\simeq pN .
\end{equation}
In the directed case, we would get the same result for both, in-degree and out-degree, since the factors $2$ and $\frac{1}{2}$ would just disappear in \eqref{eq:mean_er_degree}.
Equation \eqref{eq:mean_er_degree} demonstrates that the system behavior for each value of $p$ depends on the system size.
We choose the mean degree as a convenient parameter for the analysis of random graphs, since it can be used to replace the explicit system size.

We obtain the \emph{degree distribution} of $G_{N,p}$, if we realize that the probability to find a node with degree $k$ is equal to the probability to find a node that is connected to $k$ other nodes, but not to the $N-k-1$ remaining nodes in the network.
Thus, the degree distribution is immediately given by a bimodal distribution
\begin{equation}\label{eq:bimodal}
P(k)= \left(\begin{array}{c}N-1 \\k\end{array}\right) p^k (1-p)^{N-k-1} .
\end{equation}
Provided that we are interested in large networks ($N\rightarrow \infty $), Equation~\eqref{eq:bimodal} can be approximated by a Poisson distribution,
\begin{equation}\label{eq:poisson_distribution}
P(k) =\frac{\mean{k}^k }{k!} e^{-\mean{k}}
\end{equation}
i.e. there is variation in the degrees, but there still remains a \emph{typical degree} in the system.

It is an interesting feature of random graphs that for different edge occupation probabilities they show different phases.
For low values of $p$, nodes tend to form small connected components, whereas for increasing $p$ a \emph{giant component} emerges.
The giant component contains the majority of all nodes of the network.
The behavior for large values of $p$ has first been studied by Erd\H{o}s and R\'enyi \citep{ER:1959}.
One year later, Erd\H{o}s and R\'enyi found thresholds for the emergence of subgraphs and a giant connected component \citep{ER:1960,ER:1961}.
Their results for the occurrence of different subgraphs are summarized in \citep{RevModPhys.74}.

%
\begin{SCfigure}
\includegraphics{ER_emergence_LCC.pdf}
\caption{Emergence of the largest connected component (LCC) in an \ER graph as it follows from \eqref{eq:giant_component}
The size of the of the largest component takes finite values for $\mean{k}>1$.
The mean cluster size is given by Equation~\eqref{eq:mean_cluster_size} and diverges at $\mean{k}=1$.
}
\label{fig:ER_lcc_emergence}
\end{SCfigure}
%
The size of the giant component and the mean component size can be computed analytically for random networks.
Following \citeauthor{Newman2003}, we observe that the probability that a node is not in the giant component is equivalent to the probability that none of its neighbors is part of the giant component \citep{Newman2003}.
If $u$ is the fraction of nodes that are not in the giant component, this probability is given by $u^k$.
An expression for $u$ can be obtained by averaging $u^k$ over all degrees $k$.
The degree distribution is given by \eqref{eq:poisson_distribution}.
Hence, the fraction of nodes not in the giant component is
\[
u=e^{\mean{k}(u-1)}.
\]
The size of the giant component is $S=1-u$ and consequently
\begin{equation}\label{eq:giant_component}
S=1-e^{-\mean{k}S}.
\end{equation}
One can use similar arguments to obtain an expression for the mean cluster size \citep{Newman2003}
\begin{equation}\label{eq:mean_cluster_size}
\mean{s}=\frac{1}{1-\mean{k}+\mean{k}S}.
\end{equation}
%
The mean cluster size \eqref{eq:mean_cluster_size} and a numerical solution of Equation~\eqref{eq:giant_component}
are shown in Figure~\ref{fig:ER_lcc_emergence}.
As the figure demonstrates, the system shows a second-order phase transition at $\mean{k}=1$.
%The largest connected component phase transition is shown in figure~\ref{fig:ER_lcc_emergence}.

Since all edges in a random network are independent and identically distributed, the probability that a given node is part of a connected triple is $p^2$.
In analogy, the probability that a given node belongs to a closed triangle is $p^3$.
Consequently, the \emph{clustering coefficient}  \eqref{eq:clustering_coefficient} of a $G_{N,p}$ network is given by
\begin{equation}\label{eq:er_clustering}
C=\frac{p^3}{p^2}=p=\frac{\mean{k}}{N}.
\end{equation}
Equation \eqref{eq:er_clustering} implies that the clustering coefficient of random graphs vanishes in the limit of large networks.
%
\begin{SCfigure}
\includegraphics{ER_shortest_path_histogram.pdf}
\caption{Shortest path length distribution for a realization of a directed \ER network of the ensemble $G_{N,p}$ for $N=1000$ and $p=0.002$.
Equation \eqref{eq:er_av_shortest} gives a mean value of $8.18$, while the computed value is $9.08$.
The discrepancy vanishes in the limit of infinite graphs $N\rightarrow \infty$.
The maximum shortest path length is $18$ in this example.
It defines the diameter of the network.}
\label{fig:ER_shortest_path_histogram}
\end{SCfigure}
%

We end this section by giving an approximation of the average shortest path distance in random graphs. 
Starting at some node in the network, the average number of nodes at distance 1 is given by the mean degree $\mean{k}$. 
Hence, the average number of neighbors at distance $d$ is $\mean{k}^d$.
In order to reach all $N$ nodes in the network, we need $r$ steps, where $r$ is determined by $\mean{k}^r\simeq N$.
Thus, $r$ approximates the diameter of the network.
Since we are only interested in the rough behavior of the average shortest path length $\mean{l}$, we approximate it by $r$ \citep{dynamical_processes} and obtain 
\begin{equation}\label{eq:er_av_shortest}
\mean{l}\simeq \frac{\log N}{\log \mean{k}}.
\end{equation}
The average degree remains constant for different network orders, so that Equation~\eqref{eq:er_av_shortest} demonstrates that the average shortest path length grows logarithmically with the number of nodes in \ER graphs.
Figure \ref{fig:ER_shortest_path_histogram} shows the shortest path length distribution for one realization in the $G_{N,p}$ ensemble.
Note that the mean value $\sim 10$ is relatively small compared with the number of nodes in the network (1000).
This relation is found in many complex networks and is an indication for the small-world effect (see Section~\ref{sec:watts_strogatz_model}).

\subsection{Watts-Strogatz model}\label{sec:watts_strogatz_model}
We have seen that random graphs can reproduce some important properties of real-world networks, particularly the existence of a giant component and small average shortest path length.
Nevertheless, Equation~\eqref{eq:er_clustering} demonstrates that the tendency to form connected triangles is absent in \ER networks.
Observations show, however, that many real-world networks exhibit this feature \citep{Milgram:1967,WassermanFaust,Newman2003}.
It is characteristic for social networks in particular to have a high degree of clustering and at the same time short-cuts allowing for small average shortest path lengths.
In this sense they can be seen as an intermediate structure between lattices (high local order) and random graphs (small shortest path lengths).
Therefore, \citeauthor{Watts:1998} introduced the \emph{small-world model} in 1998 \citep{Watts:1998}.
We briefly summarize some of their main findings.
%
\begin{SCfigure}
\includegraphics{Watts_strogatz.pdf}
\caption{Clustering coefficient and average shortest path length in the Watts-Strogatz model.
Both quantities are  normalized to the the corresponding value for $p=0$.
Results for networks with $N=1000$ nodes and $m=10$.
Every data point is the average of 1000 realizations.}
\label{fig:watts_strogatz}
\end{SCfigure}

A Watts-Strogatz network interpolates between lattices and random networks by rewiring edges of a lattice.
We start with a regular ring lattice of $N$ nodes, where each node is connected to $m$ of its nearest neighbors.
Then, each edge is rewired randomly with probability $p$.
Keeping $m$ constant from the beginning yields a scalable topology for different values of $p$.
The clustering coefficient $C$ and the average shortest path length $\mean{l}$ for different values of $p$ are shown in Figure~\ref{fig:watts_strogatz}.
Both values are normalized by their corresponding values in the initial lattice, i.e. $C/C_0$ and $\mean{l}/\mean{l}_0$ respectively.

The degree distribution collapses to a single peak for $p=0$.
In their paper on the properties of small-world networks \citep{Barrat:2000fj}, the authors showed that the degree distribution converges to a Poisson distribution in the limit $p\rightarrow 1$ and found an analytical approximation for the clustering coefficient for different values of $p$.
The percolation threshold of small-world networks was investigated in \citep{Ball:1997tq,Sander2002293}, where the authors found the threshold to be reduced for increasing values of $p$.

There is no sharp criterion for a network to be called small-world network.
Instead, a network is called small-world network, if it shows a sufficiently large clustering coefficient \emph{and} a sufficiently low average shortest path length.
This is the intermediate region in Figure~\ref{fig:watts_strogatz}. 

\subsection{Barab\'asi-Albert model}\label{sec:BA_model}
Besides the critical behavior in \ER networks and the small-word effect in Watts-Strogatz networks, observations of real networks showed that they possess heavy-tailed degree distributions \citep{Barabasi99,Liljeros:2001p841}.
A central question is, where such distributions originate from.
Therefore, \citeauthor{Barabasi99} introduced a network model in order to mimic the evolution of the world-wide-web \citep{Barabasi99}.
%
\begin{SCfigure}
\includegraphics{BA_graph.pdf}
\caption{Cumulative degree distribution of a \BA graph with $N=10^5$ nodes and $m_0=m=5$.
The dashed line shows a power-law $P(k)\propto k^{-2}$.}
\label{fig:BA_cdf}
\end{SCfigure}
%
The system under consideration is a network of websites that are connected by hyperlinks and should not be confused with the physical network of internet routers.
The evolution of the www-network is reduced to two simple principles.
(1) new nodes are added to the system over time and (2) the new nodes have a higher probability to link to existing nodes of higher degree.
The second principle can be summarized as a rich-get-richer phenomenon, i.e. the more links you have the more you will get.
In network language, this mechanism is called \emph{preferential attachment}.
It can be seen as the network version of what is also known as Matthew-effect or cumulative advantage \citep{Merton:1968fh,price:1976}.

The preferential attachment model for growing networks is as follows:
Start with a small number $m_0$ of nodes and add a new node at every time step.
Connect the new node to $m<m_0$ existing nodes, each with probability $\Pi $.
Thus, $m=1$ yields a tree and $m>1$ gives a graph with cycles.
The probability for an existing node $i$ to be connected with the new one depends on the degree of $i$, i.e. $\Pi (k_i)=k_i/\sum _j k_j$.

Figure \ref{fig:BA_cdf} shows the degree distribution of a network generated this way.
We have to point out that it is generally more appropriate to plot the cumulative distribution of such distributions, because it is more robust against statistical fluctuations, particularly in the tail of the distribution \citep{Clauset:2009}.
As the figure shows, the distribution is well approximated by a power law of the form
\[
P(k)\propto k^{-\xi }
\]
with $\xi =2$ for the cumulative distribution and $\xi = 3$ for the probability density function, respectively.

Barab\'asi and Albert could show analytically that the resulting network has a power-law degree distribution of the form
\begin{equation}\label{eq:BA_law}
P(k)=2m^2 k^{-3}.
\end{equation}
Although the slope $\xi = 3$ does not match the power-law exponent of the world-wide web ($\xi = 2.1\pm 0.1$ \citep{Barabasi99}) the model explains the existence of a scale-free degree distribution.

Being a conceptional model, the \BA model is extensively used for the investigation of theoretical questions.
In fact, the power-law degree behavior is also reproduced by fitness models \citep{Bianconi:2001,Fortunato:2006} and copy models \citep{Kleinberg99theweb}.
Fitness models allow for higher flexibility in terms of the power-law exponent.
However, the range of possible exponents cannot take values in the interval $0<\xi < 2$ \citep{all_scale_free_are_sparse}.

Besides the models discussed above, there are other network models, such as the configuration model or exponential network models.
The \BA model can be extended a redirection algorithm in order to obtain other scaling exponents \citep{Krapivsky:2001bg}.
A model which is focused on real world data is the \emph{configuration model}, a more sophisticated random graph model that allows for arbitrary degree distributions \citep{Newman:2001pa,Newman:book}.
Moreover, the degree sequence of a given network remains constant.
In the configuration model one can consider higher order statistics, such as degree correlations and the clustering coefficient.
\emph{Exponential random graphs} are related to the concept of the micro canonical ensemble in statistical mechanics \citep{Strauss:86}.
In this context, an \ER graph is just one realization of an ensemble of possible random graphs.
Exponential random graphs are an elegant way of treating networks, but their mathematical treatment appears intractable for many cases of interest \citep{Newman2003}.

\subsection{Resilience of different network types}\label{sec:resilience}
A fundamental difference between complex networks and man made technological systems is their topologically induced robustness against failure.
Failure can be modeled by \emph{randomly} removing nodes of the system\footnote{Removing edges instead of nodes gives similar results.}.
In this sense, network failure can be seen as an inverse percolation problem.
The degree of failure is then given by the fraction of removed nodes $f$ and the sensitivity of a network to random failure can be measured in terms of the size of its largest connected component, which is inherently related to its functionality.
As an example, if only a few circuits in a computer would randomly fail, the largest connected component would disintegrate into smaller circuits and the machine is likely to malfunction.
It is characteristic for complex networks, however, that randomly removing nodes does not drastically change the connectivity of the network.
\citeauthor{Albert:2000} have measures the effect of network failure for different network types in \citep{Albert:2000}.
The authors found that \ER networks are more prone to random failure than scale-free networks.
The robustness of scale-free networks against random node removal is explained by the huge number of low-degree nodes in the network, so that it is unlikely to remove a hub at random.
%
\begin{SCfigure}
\includegraphics{Robustness.pdf}
\caption{Robustness of a \BA  (BA) network and an \ER  (ER) graph to random failure (grey dashed line) and targeted attack (red).
Red lines represent the size of the largest connected component (LCC) under targeted removal of the most connected nodes.
The size of the LCC remains finite for the \BA network under random failure even for a large number of removed nodes.
}
\label{fig:robustness}
\end{SCfigure}
%

The situation changes dramatically, when nodes are not removed at random, but targeted, i.e. the most central nodes are removed first.
This procedure models targeted \emph{attacks} on the network.
\citeauthor{Albert:2000} found that scale-free network are extremely vulnerable to attack of the most central nodes.
Figure \ref{fig:robustness} shows the size of the largest connected component (LCC) vs. the fraction of removed nodes for an \ER network and a scale-free \BA graph.
The figure shows results for a \BA network with $m=2$ and a \ER network with $p=0.0004$ at the beginning.
Both networks have $10^4$ nodes.
Note that the \BA network does not show a finite threshold for random node removal as the \ER network.
Thus, the network shows finite connected components even if a very large number of nodes has been removed.
The robustness against random removal comes at the price of high vulnerability against removal of the most connected nodes (red lines).
After removing a relatively small fraction of high-degree nodes, the \BA network disintegrates into small components.

A different measure of integrity of a network is how the diameter changes when nodes are removed at random or after a certain criterion.
The differences between random and scale-free networks remain similar in this perspective.
In addition, the definition of a targeted attack can be extended to any centrality measure.
Although many centrality measures correlate in many network models \citep{dynamical_processes}, different attack strategies may be effective in real networks \citep{holme:2002}.

\subsection{Epidemics on networks}\label{sec:epi_networks}
The spread of infectious diseases on networks is substantially related to network resilience.
As we have seen in Section~\ref{sec:sir_model}, individuals are removed from the population in an SIR-type disease.
This corresponds to the failure of nodes as discussed previously.
Moreover, results from attacking networks can be carried over to vaccination strategies.
The central subjects of interest remain the same as in Section~\ref{sec:sir_model}, namely the epidemic threshold $R_0$ and the outbreak size $R_\infty $.

We have seen in sections \ref{sec:si_model} and \ref{sec:sir_model} how epidemics can be modeled under the assumption of homogeneous mixing of individuals.
Nevertheless, data sources are available allowing for a more detailed analysis of an epidemic spreading process.
We start by considering the network models as introduced in Section~\ref{sec:network_theory} and review some results about the impact of different topologies on spreading processes.

\paragraph{Epidemic models on homogeneous contact networks\color{Cayenne}{.}}
To begin with, we consider a 2-compartment SI-model on a network of $N$ individuals, where a fraction $i(t)=I(t)/N$ individuals are infected and the remaining fraction $s(t)=1-i(t)$ is susceptible.
The force of infection (\eqref{eq:force_of_infection} in Section~\ref{sec:force_of_infection}) models the effective interaction between susceptible and infected individuals in terms of passing on the infection.
In a homogeneous network, e.g. an \ER or Watts-Strogatz network, the force of infection is $\lambda = \beta ki$, where $ki$ is the number of infectious contacts for a node of degree $k$ and $\beta $ is the probability of infection per time unit \citep{dynamical_processes}.
Consequently, $1/\beta $ is the spreading time scale of the process.

In order to obtain a rate-equation for the total number of infected in a homogeneous network, we replace the local degree $k$ by the mean degree $\mean{k}$ and obtain
\begin{equation}\label{eq:si_network}
\frac{di(t)}{dt}=\beta \mean{k} i(t) [1-i(t)],
\end{equation}
where $1-i(t)$ is the fraction of susceptible nodes.
This model can easily be extended to a SIS model by adding a loss term $-\gamma i(t)$ to Equation~\eqref{eq:si_network}.
Setting $\gamma =1 $ without loss of generality, we obtain 
\begin{equation}\label{eq:sis_network}
\frac{di(t)}{dt}=-i(t)+ \beta \mean{k} i(t) [1-i(t)] .
\end{equation}
The behavior of the SIS-model has been studied for Watts-Strogatz and \BA networks in \citep{Pastor-Satorras_vespi:2001}.
Following \citeauthor{Pastor-Satorras_vespi:2001}, we compute the steady state of \eqref{eq:sis_network} in order to find the epidemic threshold, that is
\[
i[-1+\beta \mean{k} (1-i)]=0.
\]
$\beta $ being fixed as a local reaction constant, the average degree $\mean{k}$ remains the only parameter in this equation.
We define the critical connectivity $\beta _c=\mean{k}^{-1}$ and obtain distinct regimes for different values of $\beta $.
Thus, the density of infected in the endemic state is
\begin{align}\label{eq:endemic_ws}
i &= 0   & ~ &\mathrm{if} \quad \beta < \beta _c \nonumber \\
i &= 1-\frac{\beta _c}{\beta }  & ~ &\mathrm{if} \quad \beta > \beta _c .
\end{align}

This shows that the threshold behavior seen in Section~\ref{sec:sir_model} for homogeneously mixed populations remains unchanged for homogeneous networks.
In fact, is has been shown that homogeneously mixed epidemic models can always be mapped onto a percolation process on a regular lattice \citep{Grassberger1983157,Sander2002293}.

\paragraph{Impact of heterogeneous connectivity\color{Cayenne}{.}}
In order to consider networks with heavy-tailed degree distributions, we modify the SIS model above and include the heterogeneity of node degrees explicitly \citep{Pastor-Satorras_vespi:2001}.
\citeauthor{Pastor-Satorras_vespi:2001} replaced the infected compartment $i(t)$ by the fraction of infected with a given degree, that is $i(t)\rightarrow i_k(t)$.
The average degree in \eqref{eq:sis_network} is replaced by the actual degree and the force of infection is extended by the probability $\Theta (i(t))$ that a given link points to an infected node.
The latter depends on the total density of infected and it depends only on $\beta $ in the steady state.
This gives the following SIS model for heterogeneous networks:
\begin{equation}\label{eq:sis_het_network}
\frac{di_k(t)}{dt}=-i_k(t)+\beta k [1-i_k(t)] \Theta (i(t)).
\end{equation}
%
\begin{SCfigure}
\includegraphics{Pastor-Satorras.pdf}
\caption{Fraction of infected in the endemic state for an SIS model.
The figure reveals the disappearance of the epidemic threshold for in \BA networks (red).
The epidemic threshold remains finite (here: $\beta _c=1/6$) for homogeneous networks and $\beta _c \rightarrow 0$ for \BA networks.
}
\label{fig:pastor-sat}
\end{SCfigure}

%
\citeauthor{Pastor-Satorras_vespi:2001} found an analytic expression for the steady state by using statistical arguments to obtain an expression for $\Theta (i(t))$.
After some calculations, the density of infected in the endemic state for a \BA network with average degree $m=k/2$ reads
\begin{equation}\label{eq:endemic_ba}
i \sim e^{\frac{-2}{\mean{k}\beta }}
\end{equation}
and the condition for the epidemic threshold is \citep{pastor-sat_2}
\begin{equation}\label{eq:degree_threshold}
\beta _c=\frac{\mean{k}}{\mean{k^2}} .
\end{equation}
A graphical comparison between \eqref{eq:endemic_ws} and \eqref{eq:endemic_ba} is given in Figure~\ref{fig:pastor-sat}.
It is an important result that the epidemic threshold vanishes in \BA networks.
As a consequence, random vaccination in \BA networks does not suppress a disease outbreak \citep{Keeling:2005}.
Nevertheless, Figure~\ref{fig:pastor-sat} shows that for the outbreak size remains small for $\beta \rightarrow 0$.
Finally, the absence of the epidemic threshold is generally found in infinite scale-free networks with degree distributions $P(k)\sim k^{-\xi }$ for $2\leqslant \xi \leqslant 3$.
It should be noted that a geographically embedded network with the same degree distribution can still show a finite outbreak threshold \citep{Sander20031}.

\paragraph{Vaccination strategies\color{Cayenne}{.}}
As we have seen in the previous section, random immunization fails in scale-free networks, because it gives the same priority to low degree nodes and large hubs, while large hubs are unlikely to be chosen by chance.
Random immunization effectively reduces the infection rate $\beta \rightarrow \beta (1-g)$, where $g$ is the fraction of vaccinated nodes.
Therefore, the epidemic threshold condition \eqref{eq:degree_threshold} reads $\beta (1-g_c)=\mean{k}/\mean{k^2}$ with the critical immunization density $g_c$.
It follows that
\begin{equation}\label{eq:critical_vacc}
g_c=1- \frac{1}{\beta } \frac{\mean{k}}{\mean{k^2}}.
\end{equation}
Given a scale-free network with diverging $\mean{k^2}$, the total population would have to be vaccinated in order to drop the infection rate below the epidemic threshold.

Nevertheless, scale-free networks are vulnerable to targeted removal of highly connected nodes, as we have seen in Section~\ref{sec:resilience}.
Immunization of the mostly connected nodes is therefore an effective vaccination strategy on these networks.
Numerical results for different vaccination strategies applied to a SIS-disease in a \BA network are shown in Figure~\ref{fig:targeted_immunization}.
%
\begin{SCfigure}
\includegraphics{SIS-Vacc.pdf}
\caption{Targeted and random vaccination for an SIS-type disease in a \BA network with $10^5$ nodes and $m=4$.
Infection parameters $\beta / \mu =2$.}
\label{fig:targeted_immunization}
\end{SCfigure}
%

In analogy to \eqref{eq:critical_vacc}, an analytic expression for the critical immunization density can be computed also for heterogeneous networks \citep{PastorSat:immunization}.
In this case, the fraction $g$ of nodes with the highest degrees in the network is vaccinated.
This introduces a cut-off degree $k_c(g)$ so that all nodes with degree $k>k_c$ do not contribute to the spread of the disease.
For the case of a \BA network \citeauthor{PastorSat:immunization} found an expression for the critical vaccination density to be
\begin{equation}\label{eq:crit_vacc_BA}
g_c \sim \exp (-2\mu / m\beta ),
\end{equation}
where $m$ is the minimum degree of the network and $\mu $ and $\alpha $ are infection parameters, respectively.
The exact value of $g_c$ can be found by extrapolation of the curves in Figure~\ref{fig:targeted_immunization}.
The striking feature of Equation~\eqref{eq:crit_vacc_BA} is, however, that the fraction of nodes that have to be vaccinated decreases exponentially with the spreading rate.

Besides the degree, we have to point out that any centrality measure (see Section~\ref{sec:micro_measures}) can be used in order to define a ranking of nodes.
This node ranking can then be used to define a vaccination priority for all nodes.
A generalized node ranking approach is of particular interest for networks, where the degree is not correlated to other centrality measures, as for example found in \citep{Guimera:2005p5232}.
A betweenness-based vaccination has been proposed in \citep{holme:2002}.

It should be noted that global knowledge about the network structure is needed in order to apply vaccination strategies as degree targeted vaccination.
However, the detailed contact structure of many real systems -- especially human contacts -- is not known.
Targeted immunization as described above can therefore be considered as an ideal vaccination strategy.
This ideal strategy can be approximated using \emph{nearest neighbor vaccination} \citep{Cohen:PRL}.
The basic idea is to use local information by just asking for the neighbors of an individual, which gives some edges of the network.
It is generally more probable that a randomly chosen edge is connected to a node of large degree, simply because these node class is connected to relatively many edges.

\paragraph{Metapopulations\color{Cayenne}{.}}
The models and results discussed so far considered every node in the network as one individual.
In many systems, however, the detailed internal contact structure is unknown, but information about contacts between whole subpopulations is available.
A subpopulation can be a city in a mobility network, an agricultural holding in a livestock trade network or a habitat in ecology. 
A \emph{metapopulation} is a set of subpopulations which are connected by migration processes \citep{Grenfell:1997ts,Hanski:1998,dynamical_processes}.
Recent works have made use of metapopulation approaches to model large scale disease outbreaks \citep{VittoriaColizza02142006}, such as influenza \citep{DuyguBalcan12222009} and SARS \citep{Hufnagel:2004p27}.

The computation of outbreak thresholds in metapopulations was addressed in \citep{Colizza:2007p6276,Colizza:2007p5906} and the spreading velocity was additionally analyzed in \citep{Belik:2011ke}.
The impact of network topology on disease spread in metapopulations was addressed in (\citet{Lentz:2012pre}, Section~\ref{sec:PRE}).
Although metapopulation approaches provide a useful tool for the modeling of epidemics, they systematically overestimate the outbreak size when compared to individual resolved approaches \citep{Keeling11052010}.

\begin{SCfigure}
\includegraphics{Metapop_scheme.pdf}
\caption{Three metapopulations $\mu $,$\nu $ and $\sigma $ of different size and infection status.
The infection status is represented by the local color distribution.
The edge $(\mu, \nu )$ indicates migration from $\mu $ to $\nu $.}
\label{fig:metapop_scheme}
\end{SCfigure}

In the context of epidemics every subpopulation has a different infection status, i.e. a distribution of $S$, $I$ and $R$.
Additionally to the local infection model, we add a migration term so that the general form of a metapopulation SIR-infection-model for a subpopulation $\mu $ is
\begin{equation}\label{eq:general_metapop_sir}
\frac{dI_\mu }{dt} = R(S_\mu, I_\mu , R_\mu) + M (S_\mu, I_\mu , R_\mu, S_\nu ,I_\nu ,R_\nu, \tau ).
\end{equation}
The first term $R$ in Equation~\eqref{eq:general_metapop_sir} is a \emph{local reaction} term, while the \emph{migration} $M$ to other subpopulations could depend on the local distribution and the infection status of other subpopulations connected to $\mu $.
Furthermore, the migration between subpopulations could occur on a time-scale $\tau $ different from the time-scale of the local infection.
The impact of these time-scales on disease spread was analyzed in  \citep{cross2005,Balcan:2011gv,Lentz:2012pre}.
We investigate the interplay between network properties and disease outbreaks in Section~\ref{sec:PRE}. 
%


%\bibliographystyle{apalike}
%\bibliography{/Users/lentz/Documents/GitHub_locals/Thesis/bibliography.bib}
%\end{document}

        \chapter{Outlook}
We give a description of temporal networks later.



        % Part 2
        \part{Static Networks}

\chapter{Basics}



\chapter{Advanced}



        
        % Part 3
        %\part{Temporal networks}
\chapter{Temporal network analysis}
\texttt{In the previous chapter, we used a dataset covering the period 01 June 2006 -- 31 December 2008.
The temporal network chapter uses data 01 January 2008 -- 31 December 2009.
Since data is then congruent with the publications.}

The previous chapter has demonstrated that network analysis provides a deep insight into the processes behind epidemic spreading.
Given a sufficient amount of data, a contact network is capable to capture all possible infection pathways in the system.
The potential of static network analysis lies in the huge toolbox of methods that has been developed in the last decades.
As depicted in section \ref{sec:network_theory}, there exist coherent definitions for both their large scale topological features and local centrality measures allowing for node rankings.

Nevertheless, the concept of static networks neglects temporal variations in the system, i.e. the edges of a particular network are not necessarily present all the time.
This chapter addresses some of the conceptional problems owing to a sparse and heterogenous occurrence of edges in the network, the most central one being the \emph{causality of paths} in the network.
Section XX focusses on the computational analysis of the full temporal representation of the network analyzed in section \ref{sec:network_analysis}.
In section XX, we present a novel formalism mapping the causality of temporal networks onto a mathematical graph.


\section{Introduction}


\section{Data driven network analysis}

\subsection{Representative sample}

\subsection{Node rankings}

\subsection{Temporal vs. static representation}

\section{Formalism driven network analysis}

\subsection{Matrices for temporal networks}

\subsection{Representative sample / characteristic time scale}

\subsection{Causal fidelity}

\subsection{Temporal and topological mixing patterns}

\subsubsection{Randomized models}








        %\part{nix}
%\chapter{Introduction}
\section{Perron-Frobenius Theorem for $\mathbf{P}_n$}
In the static case, the matrix $\mathbf{A}+\mathbf{1}$ is always primitive, i.e. if $\mathbf{A}$ is the adjacency matrix of a connected graph, than $(\mathbf{A}+\mathbf{1})^N$ is always full for one $N$.
(This does not necessarily hold for $\mathbf{A}^N$ alone).
What follows from this statement is, that $\mathbf{P}_n$ should yield a proper eigenvector centrality.

\section{Conceptional problems with components in temporal networks}
Accessibility allows for a macroscopic network view.
A hyper graph $H$ containing time labelled node triples, i.e. $(x,y,z,t_1,t_2)$ could allow for the detection of components.
The hyper edges correspond to the transitive edges of $\mathcal{G}$.
The time stamp $t_1$ marks the occurrence time of edge $(x,y)$, while $t_2$ is the occurrence time of $(y,z)$.
Therefore, the time of $(y,z)$ is always equivalent to $t_2$ and $t_1<t_2$.
%
\begin{figure}[htbp]
\begin{center}
\includegraphics{images/transitivity_scheme}
\caption{Overlapping transitive hyperedges.}
\label{fig:overlapping_hyperedges}
\end{center}
\end{figure}

The idea of detecting components is now to traverse the transitive part of $\mathcal{P}$.
Considering two overlapping hyper edges $e_1=(i,j,k,t_1,t_2)$ and $e_2=(j,k,l,t_3,t_4)$, a path $(e_1,e_2)$ is causal, if $t_3\geq t_2$ and $t_4>t_2$.
As a matter of fact, this is again a transitivity condition on $\mathcal{H}$.

The approach could be simplified, if we consider a temporal ordered sequence of the hyper graphs $\mathcal{H}$.
Every causal path then follows adjacent edges of successive snapshots.




    \backmatter
        \appendix
        %%\documentclass[openright,twoside,headsepline]{scrbook}
%\usepackage[applemac]{inputenc}
%\usepackage{graphicx,xcolor,hyperref} % obsolete in HU-diss
%\usepackage[round,authoryear]{natbib}
%\setlength\bibhang{2em} 
%
%
%\KOMAoptions{numbers=noenddot}
\usepackage{amsmath,amssymb,amsfonts,amsthm,epigraph,scrpage2}
\usepackage[ngerman,english]{babel}
\usepackage{centernot} % f�r Durchstreichung

\usepackage{currvita}
\renewcommand*{\cvheadingfont}{\large\bfseries} % CV Titel
\renewcommand*{\cvlistheadingfont}{\bfseries\sffamily} % sub-�berschriften
\renewcommand*{\cvlabelfont}{\sffamily} % items

\definecolor{Cayenne}{rgb}{0.502,0.0,0.0}
\definecolor{Steel}{rgb}{0.4,0.4,0.4}
\definecolor{Tri_blue}{rgb}{0.039,0.5098,0.8}
\definecolor{Tri_yellow}{rgb}{0.5529,0.5451,0.0549}


%\setcounter{secnumdepth}{3} % sub subsections numbering
%\setcounter{tocdepth}{3} % subsubsections inTOC

\usepackage[format=plain,singlelinecheck=false, font={sf,small},labelfont={bf,color=Steel}]{caption}
\DeclareCaptionLabelSeparator{cayenne_period}{\textcolor{Cayenne}{.} }
\captionsetup{labelsep=cayenne_period}

% Colors
\addtokomafont{chapter}{\color{Steel}}
\addtokomafont{section}{\color{Steel}}
\addtokomafont{subsection}{\color{Steel}}
\addtokomafont{subsubsection}{\color{Steel}}
\addtokomafont{paragraph}{\color{Steel}}

\addtokomafont{pagehead}{\color{Steel}}
\renewcommand{\pnumfont}{\color{Steel}} 
\addtokomafont{headsepline}{\color{Steel}} 
\pagestyle{scrheadings}

%  labels in description environments
\renewcommand{\descriptionlabel}{\hspace\labelsep{}\sffamily\small\bfseries{}\color{Steel}{}}

%\makeatletter % dot after sections and all below
%\let\std@sect\@sect
%\def\@sect#1#2#3#4#5#6[#7]#8{\std@sect{#1}{#2}{#3}{#4}{#5}{#6}[#7.]{#8\color{Cayenne}{.}}}
%\makeatother
\usepackage{acronym}

\usepackage[leftcaption]{sidecap} % inner, outer,left,right
\sidecaptionvpos{figure}{t}

% Papiergr��e
%\setlength{\paperwidth}{21cm}
%\setlength{\paperheight}{25cm}
%\recalctypearea
%\usepackage{geometry}
%\usepackage[cross,a4,center]{crop}

%% Flattersatz
%\usepackage[document]{ragged2e} % Flattersatz
%\setlength{\RaggedRightParindent}{1em} % evtl. parskip


%% Sans Serif
%\usepackage{cmbright}
%\renewcommand{\familydefault}{\sfdefault}
%% Palatino
%\usepackage[sc]{mathpazo}
%\linespread{1.05}         % Palatino needs more leading (space between lines)
%\setkomafont{sectioning}{\normalcolor\bfseries} % Kapitel�berschriften

%%% Kapitel�berschriften: Mit gro�en Zahlen
%\usepackage{titlesec}
%\titleformat{\chapter}[display]
%{\bfseries\Large}
%{ %\Huge\textsc{\chaptertitlename} % f�r das Wort 'Kapitel'
%\hfill\fontsize{120}{70}\selectfont\color{lightgray}\textbf{\thechapter}}
%{-2ex}
%%{\filleft\fontsize{50}{70}\selectfont\scshape} % Kapit�lchen oder...
%{\filleft\fontsize{50}{70}\selectfont\textbf} % ...oder keine Kapit�lchen
%[\vspace{0ex}]
%
%%%% Part�berschriften
%\titleformat{\part}[display]
%{\bfseries\Large}
%{ %\Huge\textsc{\chaptertitlename} % f�r das Wort 'Kapitel'
%\hfill\fontsize{120}{70}\selectfont\color{lightgray}\textbf{\thepart}}
%{-2ex}
%{\filleft\fontsize{50}{70}\selectfont\scshape} % Kapit�lchen oder...
%%{\filleft\fontsize{50}{70}\selectfont\textbf} % ...oder keine Kapit�lchen
%[\vspace{0ex}]


\newcommand{\ER}{Erd\H{o}s-R\'enyi }
\newcommand{\BA}{Barab\'asi-Albert }
\newcommand{\mean}[1]{\left< #1 \right>}
\newcommand{\abs}[1]{\left| #1 \right|}
\newcommand{\norm}[1]{\lVert#1\rVert}
\newcommand{\mat}[1]{\mathbf{#1}}
\newcommand{\tgraph}{\mathcal{G}}

\theoremstyle{definition} % non-italic
\newtheorem{annahme}{Annahme} % braucht amsthm
\newtheorem{definition}{Definition}
\newtheorem{theorem}{Theorem}
\newtheorem{satz}{Satz}
\newtheorem{frage}{Frage}
%\input{watermarks/watermark.tex}
\DeclareMathOperator{\nnz}{nnz}

% + Graphicspath nach begin document

% aus Doi hyperref machen
%\newcommand*{\doi}[1]{\href{http://dx.doi.org/\detokenize{#1}}{doi: \detokenize{#1}}}


%
%
%
%\begin{document}
%\graphicspath{{/Users/lentz/Documents/GitHub_locals/Thesis/images/}}
%\tableofcontents

\chapter{Appendix}

\section{Network implementation}\label{sec:implementation}
In order to efficiently implement networks and their analysis on a computer, it is necessary to use appropriate data structures.
A short and transparent introduction to data structures and algorithms is in the book of \citeauthor{algorithm_design} \citep{algorithm_design}.
In this section, we review some essential data structures appropriate for network analysis and give a brief description of fundamental algorithms.
The purpose of this section is to sketch the basic ideas behind the data structures and algorithms rather than to list algorithms and source code.
For source code of data structures and algorithms, the reader is encouraged to the lecture of \citep{algorithm_design} and \citep{Merali:2010ih}.

\paragraph{Matrix implementation\color{Cayenne}{.}}
To begin with, we consider the implementation of adjacency matrices as introduced in section \ref{sec:network_matrices}.
%Matrices can be seen as a \emph{graph centric view} on the network, since they map the whole network topology onto a single object.
Adjacency matrices are by definition square matrices.
Their entries are either $0$ or $1$.
In weighted networks, their entries can take any positive value.
The number of nodes in most complex network datasets is relatively large. 
Starting with small networks (100 nodes, conference contacts \citep{isella2011}), complex networks can be gigantic ($\sim 10^9$ nodes in the case twitter tweeds \citep{Yang:2011} or the world wide web \citep{Albert:1999uu,Broder:2000ej})
Note that the sizes of adjacency matrices scales with the square of the networks size, hence adjacency matrices of these networks are intractable for straightforward computer-based matrix analyses.

Nevertheless, it is a common feature of many real-world networks that they are sparse, i.e. the vast majority of their entries are zeros\footnote{Typically, the number of edges in the network is of the same order as the number of nodes.}.
Since zeros do not contribute to matrix operations as products or additions, it is reasonable to use data structures ignoring zeros.
These data structures are called \verb"sparse matrices".
Their advantages is (1) they save much memory and (2) computations are faster, because operations with zeros involved are not executed.
Sparse matrix data structures are available in most modern computer languages (e.g. Matlab, Python: \verb"scipy" library, C/C++: \verb"boost" library).
They perform well for problems based on adjacency matrices as the computation of the degree or eigenvector centrality.
However, matrix methods are not suitable for the computation of many other network measures, such as betweenness, closeness or network navigation.

\paragraph{Graph implementation\color{Cayenne}{.}}
The drawback of matrix representations of networks is that it is rather complicated to \emph{traverse} a network using matrices.
A traversal is a procedure of the following form: start at a node, visit all of its neighbors, from each neighbor visit its neighbor and so forth, until there are no more new nodes to traverse.
This is a searching process.
Network traversal is used in many implementations of graph theoretic methods.

As an alternative implementation of the adjacency matrix the \verb"adjacency list" is a well suited data structure for network traversal.
It stores the neighbors of every node and can be implemented as linked lists.
Adjacency lists can be considered as a node centric view on the network, since they allow for a fast access to the neighborhood of each node.
Considering the example network on the left panel of Figure~\ref{fig:dfs_bfs}, the corresponding adjacency list is as follows:
\begin{align*}
1 &\rightarrow 2,3 \\
2 &\rightarrow 4 \\
3 &\rightarrow 2 \\
4 &\rightarrow 2,3 .
\end{align*}
In order to traverse the graph starting at node $1$, we choose one of the neighbors of $1$ and repeat the process until we have traversed all nodes.
One possible traversal starting at $1$ would be $1\rightarrow 3 \rightarrow 2 \rightarrow 4$.

During a traversal process, one can decide to either exploit the whole neighborhood of a node first and then traverse the next generation or choose a neighbor of every traversed node at every step.
These two essential searching processes are called breadth-first-search (BFS) and depth-first-search (DFS), respectively.
The difference between the two lies in the order of traversed nodes.
Figure \ref{fig:dfs_bfs} shows resulting search trees of the two methods.
Starting at node $1$, the traversal $1\rightarrow 3 \rightarrow 2 \rightarrow 4$ would be found using a DFS-search, while a BFS-search would yield $1\rightarrow 2 \rightarrow 3 \rightarrow 4$.
It should be noted that in general there exist multiple BFS and DFS trees for each starting node. 
%
\begin{figure}[htb]
\begin{center}
\includegraphics{DFS-BFS.pdf}
\caption{Breadth-first-search and depth-first-search trees in a directed network. Search processes are started at node $1$.}
\label{fig:dfs_bfs}
\end{center}
\end{figure}

Both search algorithms are used in a large number of algorithmic applications.
BFS is efficient to compute shortest paths in unweighted networks.
With every generation in a BFS tree, the distance from the starting node is incremented by 1, and thus the set of nodes with a certain distance from the starting node can be directly read from the BFS tree (see Figure~\ref{fig:dfs_bfs}).
Shortest paths in weighted networks can be identified using a the algorithm of Dijkstra \citep{Dijkstra:1959}.
Connected components in directed graphs can be efficiently identified using closed DFS paths \citep{algorithm_design}.

Due to the sparsity of typical adjacency matrices, networks can also be efficiently stored as \verb"edge lists".
An edge list is a list of tuples, where each tuple $(u,v)$ is an edge connecting nodes $u$ and $v$.
The edge list of the network shows in Figure~\ref{fig:dfs_bfs} is
\begin{align*}
&(1,2)\\
&(1,3)\\
&(2,4)\\
&(3,2)\\
&(4,2)\\
&(4,3) .
\end{align*}
Due to their human readable structure, edge lists are a convenient format to store networks as column wise text files.
Edge lists can also be efficiently used for edge randomization and random graph generation.

Implementations of the graph structures discussed above are for example available in the libraries \verb"networkx" (Python), \verb"igraph" (C, Python, R), \verb"Lemon" and \verb"Boost" (C++).

\paragraph{Hard problems\color{Cayenne}{.}}
The tools introduced above provide a huge and efficient toolbox for network analysis.
Nevertheless, there are still network problems, where no efficient algorithm is known for their exact solution.
In the language of complexity theory, the time to solve these problems scales with the problem size in non-polynomial time.
There are two important complexity classes of problems in computational complexity theory.
First, the class of NP-complete (NP stands for Non-deterministic Polynomial-time) problems, and second, the class of NP-hard problems.
All problems mentioned in this thesis have been proven to be NP-complete in our context.
For most practical questions, a distinction between the two classes is irrelevant, however.
It is rather important to recognize intractable graph theoretical problems.
NP complete problems can typically be solved exactly only for small system sizes.

Probably the most popular example is the \emph{traveling salesman problem}: a salesman has to traverse a set of cities and thereby choose the order of those cities that minimizes the total distance.
For small problem sizes, it is possible simply to try out all possible combinations and find the minimal total distance (brute-force search).
The number of possible combinations, however, grows factorial with the system size, i.e. finding a solution takes $t\propto n!$ for $n$ cities.
In other words, if the problem could be solved for 20 cities in 1 second, it would take 21 seconds to solve it for 21 cities, 7 minutes for 22 cities and 3 million years for 30 cities!

A more exhaustive overview about hard problems is in \citep{algorithm_design} and the references therein.
Generally, heuristic methods have to be used in order to get an approximate solution.
It should be noted that the \emph{maximum clique} problem (see section \ref{sec:PRL}) and \emph{graph partitioning} (see section \ref{sec:network_analysis}, Equation~\eqref{eq:modularity}) belong to the class of hard problems \citep{brandes2007}.

\section{Degree vs. other centrality measures}\label{sec:centrality_hit}
In this section, we compute the centrality measures introduced in Section~\ref{sec:micro_measures} and compare them with the degree.
In particular, we compare the degree with betweenness, closeness and eigenvector centrality.
Since the eigenvector centrality is defined only on connected networks, the analysis is restricted to the LSCC of the network.
%
\begin{figure}[htb]
\begin{center}
\includegraphics{static_centrality_pvm.pdf}
\caption{Correlation between the degree and other centrality measures for the livestock trade network.
The left panel shows the closeness centrality $C_C$, eigenvector centrality $C_E$ and betweenness centrality $C_B$ vs. the in-degree.
The respective picture for the out-degree is shown on the right.
Dashed lines show power-law fits of the data.
Only nodes with in/out-degree greater than 10 are shown.}
\label{fig:centrality_pvm}
\end{center}
\end{figure}
%
Figure~\ref{fig:centrality_pvm} shows different centrality measures for the network being compared with the degree.
Each point represents one node, and the dashed lines show power-law fits of the scatter plots, respectively.
The figure demonstrates that all considered centrality measures show a correlation with the the degree.

\section{Subgraphs and maximum modularity}\label{sec:maximum_modularity_subgraphs}
Although computer generated modular networks are used in many applications, the author is not aware of any systematic analysis of the maximum value of modularity depending on the number of modules.
Therefore, we derive an estimation of the maximum modularity value depending on the number of modules in the network.
The results are derived for a clique of modules, but remain unchanged for a ring of modules as this distinction is reasonable in finite systems.
In addition, the estimation is also valid for directed networks.

The modularity of a network with given modular structure can be computed using the equation
\begin{equation}\label{eq:mod_undir}
Q=\sum _i \left( e_{ii}-a_i^2 \right),
\end{equation}
where $e_{ij}$ is the fraction of edges pointing from community $i$ to community $j$.
The last term corresponds to the fraction of all edges that are connected to community $i$, i.e.
\[
a_i = \sum _j e_{ij}.
\]
Since the sum over all edge fraction has to be 1, it is $\sum _{ij}e_{ij}=1$.
If a network consists of two modules $x$ and $y$, the fraction of edges in $y$ is
\begin{equation}\label{qmaxbedingung}
y=c-x,
\end{equation}
where the constant $c<1$ is the fraction of all inner module edges.
In general, this expression is $c=\mathrm{Tr} (e)$.

\subsection{Two modules}
In the case of two communities, the fraction of inter-module-edges is uniquely determined by the fraction of inner-module-edges.

The matrix $e_{ij}$ takes the form
\[
e_{ij}=\left(\begin{array}{cc}x & \frac{1}{2}(1-x-y) \\ \frac{1}{2}(1-x-y) & y\end{array}\right) ,
\]
where $x$, $y$ are the edge fractions \emph{in} communities 1 and 2 and $\frac{1}{2}(1-x-y)$ is the fraction of edges \emph{between} communities 1 and 2.
The corresponding expression for $Q$ is.
\[
Q=x-\left( x+\frac{1-x-y}{2} \right) ^2 +y -\left( y+\frac{1-x-y}{2} \right) ^2 .
\]
This function does not possess a maximum over the total domain, but there is a maximum in the subdomain $0<x<1$, $ 0<y<1$.
Condition \eqref{qmaxbedingung} yields
\[
Q=\frac{1}{2}+2\, cx-2\, x^2-\frac{1}{2}\, c^2+c.
\]
Using condition \eqref{qmaxbedingung} restricts the function to tuples $(x,y)$, where $x+y=c$, which corresponds to a line $y=c-x$.
Thus, we are looking for the maximum along this line using the condition
\[
\frac{\partial Q}{\partial x}=2c-4x=0.
\]
It follows $x=c/2$ and the maximum condition $\partial ^2 Q/\partial x^2 =-4<0$ is satisfied.
Using \eqref{qmaxbedingung} gives the solution
\begin{equation}\label{qmax_xy}
x=\frac{c}{2} , \hspace{0.5cm} y=\frac{c}{2}.
\end{equation}
The corresponding modularity is 
\begin{align*}
Q&=\frac{c}{2}-\frac{1}{4}\left( 1+\frac{c}{2}-\frac{c}{2} \right) ^2 +\frac{c}{2}-\frac{1}{4}\left( 1+\frac{c}{2}-\frac{c}{2} \right) ^2 \\
&= c- \frac{1}{2} .
\end{align*}
The case where a maximum fraction of edges is in the modules and a minimum fraction is between modules is met, if $c\rightarrow 1$.
In this case, the modularity takes its maximum value.
The limit is
\begin{equation}\label{q2limits}
\lim _{c\rightarrow 1} x =1/2, \hspace{0.5cm}\lim _{c\rightarrow 1} y =1/2, \hspace{0.5cm} \lim _{c\rightarrow 1} Q=1/2.
\end{equation}
For the case of two modules, the maximum modularity is found for two equally sized modules of approximate size $1/2$.
The maximum modularity is then $Q=0.5$.
We consider the case of more modules below.

%
\subsection{Arbitrary number of modules}
In the case of more than two modules, all modules can have different sizes in the first place and can be connected among themselves arbitrarily.
The general module-matrix takes the form
\begin{equation}\label{eq:general_module_matrix}
e_{ij}=\left(\begin{array}{ccccc}x_1 &  & \hdots &  & d \\ & x_2 &  &  &  \\\vdots &  & \ddots &  & \vdots \\ &  &  & \ &  \\d &  & \hdots &  & x_n\end{array}\right) .
\end{equation}
All non-diagonal elements are 
\[
d= \frac{1-\text{Tr } (e) }{n(n-1) } = \frac{1-c }{n(n-1) }
\]
with $c\equiv \text{Tr } (e)=\text{const.}<1$.
Thus, the general expression for modularity is
\begin{equation} \label{eq:modumatrix}
Q= c-\sum  _i \left( \sum _j e_{ij} \right)^2 .
\end{equation}
%
We use the above expression for the non-diagonal elements $d$ and compute the expression $\sum _j e_{ij}$ in Equation~\eqref{eq:modumatrix}.
\begin{equation}\label{eq:sumeij}
\sum _j e_{ij} = e_{ii} + \sum _{j\neq i} e_{ij} = x_i + (n-1) \frac{1-c}{n(n-1)} = x_i + \frac{1-c}{n}.
\end{equation}
%
Now we insert $\sum _j e_{ij}=x_i+\frac{1-c}{n}$ in Equation~\eqref{eq:modumatrix} and after some algebra we get a general expression for the modularity for networks of the form \eqref{eq:general_module_matrix}:
\begin{equation}\label{eq:Qndim}
Q=c-\sum _i x_i^2 - \frac{1-c^2}{n} = \sum _i x_i -\sum _i x_i^2 - \frac{1-(\sum _i x_i)^2}{n}.
\end{equation}

In order to find the relevant maximum of \eqref{eq:Qndim}, its slope has to vanish along a hyperplane defined by
\begin{equation}\label{qmehrdimbedingung}
\sum _i x_i = c = \text{const.} <1.
\end{equation}
%
Since $c$ is constant, the relevant part of \eqref{eq:Qndim} for the maximum is
\begin{align}
Q_{\text{relevant}}\equiv Q_\text{r}= -\sum _{i=1}^n x_i^2&=-\sum _{i=1}^{n-1} x_i^2- \underbrace{ \left( 
c-\sum _{i=1}^{n-1} x_i \right) ^2 }_{x_n^2} \label{eq:relevant_Q}\\
&= -\sum _{i=1}^{n-1} x_i^2 - c^2 + 2c\sum _{i=1}^{n-1} x_i -\left( \sum _{i=1}^{n-1} x_i \right) ^2. \nonumber
\end{align}
Note that the sum on the right-hand side is up to $n-1$.
This effectively eliminates the last variable.
The derivative of $Q$ is
\begin{equation}
\frac{\partial Q}{\partial x_i}= \frac{\partial Q_\text{r}}{\partial x_i}=- 2\sum _{i=1}^{n-1} x_i + 2c(n-1)-2 (n-1) \sum _{i=1}^{n-1} x_i.
\end{equation}
%
In order to find a maximum, the derivative has to vanish, i.e.
\begin{align*}
0&=- 2\sum _{i=1}^{n-1} x_i + 2c(n-1)-2 (n-1) \sum _{i=1}^{n-1} x_i \\
&=- \sum _{i=1}^{n-1} x_i + c(n-1) - (n-1) \sum _{i=1}^{n-1} x_i \\
 &= cn-c-n\sum _{i=1}^{n-1} x_i +\sum _{i=1}^{n-1} x_i -\sum _{i=1}^{n-1} x_i\\
 &= cn-c-n\sum _{i=1}^{n-1} x_i .
\end{align*}
It follows
\[
\underbrace{c-\sum _{i=1}^{n-1} x_i}_{x_n} =\frac{c}{n}.
\]
Thus,
\begin{equation}
x_n=\frac{c}{n}.
\end{equation}
Hence, the maximum of $Q$ is obtained, if all modules have the same size, i.e. $x_i=\frac{c}{n} \; \forall i$.

In order to find the maximum value of $Q$, we insert the module size $x_i=c/n$ into Equation~\eqref{eq:Qndim} and get
\[
Q=c-\sum _{i=1} ^n \left( \frac{c}{n} \right) ^2 - \frac{1-c^2}{n}=c-\frac{c^2}{n}-\frac{1}{n}+\frac{c^2}{n}.
\]
Thus, it follows for dense modules
\begin{equation}\label{eq:q_max}
Q_\mathrm{max}= \lim _{c\rightarrow 1}Q=1-\frac{1}{n}.
\end{equation}
Consequently, the maximum value of $Q_\mathrm{max}$ is determined by the number of modules.
A similar result was found using probabilistic arguments in \citep{Good2010}.

\paragraph{Finite systems\color{Cayenne}{.}}
In finite systems, the minimum fraction of inter-module edges is obtained, when modules are connected to each other on a ring, each module having two nearest neighbors.
In this case we set  $e_{ij}= \frac{1}{n}(1-c)$ for $j=i+1$ and $j=i-1$ and all other elements are zero.
This yields
\begin{equation}\label{eq:ring_module_matrix}
e_{ij}=\left(\begin{array}{cccccc}
x_1 & \frac{1}{n}(1-c)&  & \hdots &  & 0 \\ 
\frac{1}{n}(1-c)& x_2 &\frac{1}{n}(1-c)&  &  &  \\
 & \frac{1}{n}(1-c) & \ddots & \ddots & & \\
\vdots &  & \ddots & \ddots & \ddots &\vdots \\
 & & & \ddots   &  x_{n-1} &\frac{1}{n}(1-c) \\
0 & &  & \hdots & \frac{1}{n}(1-c) & x_n
\end{array}
\right) .
\end{equation}
It follows immediately that $\sum _j e_{ij}=x_i+\frac{2(1-c)}{n}$, which is equivalent to \eqref{eq:sumeij} up to a factor 2.
Inserting this into Equation~\eqref{eq:modumatrix} gives a similar expression for modularity \eqref{eq:Qndim} as for the general case:
\begin{equation*}
Q=c-\sum _i x_i^2 - \frac{4(1-c)}{n} .
\end{equation*}
Since the relevant part for maximum finding is the quadratic term as in \eqref{eq:relevant_Q}, the results remain unchanged for modules along a chain and the maximum value is as above
\begin{equation}\label{eq:q_max2}
Q_\mathrm{max}=1-\frac{1}{n}.
\end{equation}

Figure~\ref{fig:q_max} shows a comparison between Equation~\eqref{eq:q_max2} and a computer simulation of a ring of modules where new modules are added to the system successively and the maximum modularity is computed. 
The edge density of each module is given by the edge occupation probability $p_\mathrm{in}=0.5$.
The figure demonstrates that Equation~\eqref{eq:q_max} gives a good approximation of the maximum value $Q_\mathrm{max}$ even for small systems.
%
\begin{SCfigure}
\includegraphics{Q_max.pdf}
\caption{Equation \eqref{eq:q_max2} (grey dashed line) reproduces the values found by numerical simulations (red circles).
In the simulations, modules are dense, directed subgraphs ($p_\mathrm{in}=0.5$) with $32$ nodes each.
Modules are connected on a ring so that the resulting graph is connected.
}
\label{fig:q_max}
\end{SCfigure}

\paragraph{Directed networks\color{Cayenne}{.}}
In analog to Equation~\eqref{eq:mod_undir} the modularity of directed networks can be written as \citep{Kao:2007}
\begin{equation}\label{eq:mod_directed}
Q=\sum _i e_{ii} - a_i ^{\text{in}} a_i ^{\text{out}}.
\end{equation}
where
\[
a_j ^\text{in}= \sum _i e_{ij} \hspace{1cm} \text{and} \hspace{1cm} a_i ^\text{out}= \sum _j e_{ij}.
\]
The structure of the inter-module edges takes the form of the matrix \eqref{eq:ring_module_matrix} and thus results do not differ either for the directed case.






%
%\bibliographystyle{apalike}
%\bibliography{/Users/lentz/Documents/GitHub_locals/Thesis/bibliography.bib}
%\end{document}
%

        \nocite{*}                
		% See the documentation or under
		% http://edoc.hu-berlin.de/e_autoren/latex/bedingung.php
		% for the list of the permitted styles.
        \bibliographystyle{plainnat}
        \bibliography{bibliography}
        %\listoffigures
        %\listoftables    
        %% You can change this text, if needed.
\chapter*{Selbst�ndigkeitserkl�rung}
\selectlanguage{ngerman}
Ich erkl�re, dass ich die vorliegende Arbeit selbst�ndig und nur unter Verwendung der angegebenen Literatur und Hilfsmittel angefertigt habe.

\vspace{2\baselineskip}
\noindent Berlin, den \today\hfill\authorfirstname \authorsurname
\selectlanguage{english}
    % Use our template or write your own.
\end{document}
