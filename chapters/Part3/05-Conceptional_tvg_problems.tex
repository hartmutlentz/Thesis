%\part{nix}
%\chapter{Introduction}
\section{Perron-Frobenius Theorem for $\mathbf{P}_n$}
In the static case, the matrix $\mathbf{A}+\mathbf{1}$ is always primitive, i.e. if $\mathbf{A}$ is the adjacency matrix of a connected graph, than $(\mathbf{A}+\mathbf{1})^N$ is always full for one $N$.
(This does not necessarily hold for $\mathbf{A}^N$ alone).
What follows from this statement is, that $\mathbf{P}_n$ should yield a proper eigenvector centrality.

\section{Conceptional problems with components in temporal networks}
Accessibility allows for a macroscopic network view.
A hyper graph $H$ containing time labelled node triples, i.e. $(x,y,z,t_1,t_2)$ could allow for the detection of components.
The hyper edges correspond to the transitive edges of $\mathcal{G}$.
The time stamp $t_1$ marks the occurrence time of edge $(x,y)$, while $t_2$ is the occurrence time of $(y,z)$.
Therefore, the time of $(y,z)$ is always equivalent to $t_2$ and $t_1<t_2$.
%
\begin{figure}[htbp]
\begin{center}
\includegraphics{images/transitivity_scheme}
\caption{Overlapping transitive hyperedges.}
\label{fig:overlapping_hyperedges}
\end{center}
\end{figure}

The idea of detecting components is now to traverse the transitive part of $\mathcal{P}$.
Considering two overlapping hyper edges $e_1=(i,j,k,t_1,t_2)$ and $e_2=(j,k,l,t_3,t_4)$, a path $(e_1,e_2)$ is causal, if $t_3\geq t_2$ and $t_4>t_2$.
As a matter of fact, this is again a transitivity condition on $\mathcal{H}$.

The approach could be simplified, if we consider a temporal ordered sequence of the hyper graphs $\mathcal{H}$.
Every causal path then follows adjacent edges of successive snapshots.


