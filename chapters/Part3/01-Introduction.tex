%\part{Temporal networks}
\chapter{Temporal network analysis}
\texttt{In the previous chapter, we used a dataset covering the period 01 June 2006 -- 31 December 2008.
The temporal network chapter uses data 01 January 2008 -- 31 December 2009.
Since data is then congruent with the publications.}

The previous chapter has demonstrated that network analysis provides a deep insight into the processes behind epidemic spreading.
Given a sufficient amount of data, a contact network is capable to capture all possible infection pathways in the system.
The potential of static network analysis lies in the huge toolbox of methods that has been developed in the last decades.
As depicted in section \ref{sec:network_theory}, there exist coherent definitions for both their large scale topological features and local centrality measures allowing for node rankings.

Nevertheless, the concept of static networks neglects temporal variations in the system, i.e. the edges of a particular network are not necessarily present all the time.
This chapter addresses some of the conceptional problems owing to a sparse and heterogenous occurrence of edges in the network, the most central one being the \emph{causality of paths} in the network.
Section XX focusses on the computational analysis of the full temporal representation of the network analyzed in section \ref{sec:network_analysis}.
In section XX, we present a novel formalism mapping the causality of temporal networks onto a mathematical graph.


\section{Introduction}


\section{Data driven network analysis}

\subsection{Representative sample}

\subsection{Node rankings}

\subsection{Temporal vs. static representation}

\section{Formalism driven network analysis}

\subsection{Matrices for temporal networks}

\subsection{Representative sample / characteristic time scale}

\subsection{Causal fidelity}

\subsection{Temporal and topological mixing patterns}

\subsubsection{Randomized models}







