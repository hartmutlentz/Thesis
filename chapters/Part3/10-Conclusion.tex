\chapter{Conclusion}
In this thesis we have examined the role of paths for the spread of infectious diseases on networks.
The importance of paths in the context of epidemiology has been demonstrated for the case of static networks and was then carried over to the temporal case.
As a central result, we have introduced the unfolding accessibility method of temporal networks in order to analyze the path structure of these systems.

Concerning the spread of infectious diseases on complex networks, detailed knowledge of the dynamical aspects of disease transmission is not known in most real-world scenarios.
It turns out however that the pure topology of contact patterns is of major importance in this context.
In contrast to the infection parameters of most diseases, the contact structures can be measured to great detail for a large number of real-world systems.
Although these contact structures form complex networks, which exhibit a variety of properties in most examples, it turns out that the structure of paths in these systems defines the domain for any spreading process.
On the one hand, the range (out-component) of a node defines an upper limit for the size of disease outbreaks.
On the other hand, the path structure of the whole system can be mapped onto the accessibility (transitive closure) of the network.

In this thesis, we have analyzed the impact of two particular attributes of \emph{static} complex networks on the properties of their paths.
As a generic complex network, we analyzed the properties of a livestock trade network in Germany in detail.
Among other features, the network exhibits a giant component and a significant modular structure.
We have seen that the giant component has a striking effect on the spreading potential of the network nodes.
Whenever a network is close to its critical point, its nodes can be divided into 2 disjoint classes.
A weaker restriction on the path-connectivity between nodes is the concept of modules, i.e. densely connected subgraphs being sparsely interconnected.
These structures have a relatively weak effect on the outbreak size.
This is particularly true for meta population networks, where nodes are permeable for disease spread, if they are not fully recovered.
Nevertheless, a modular network is likely to show a significantly delayed outbreak peak.
This result can be useful for the implementation of counter measures, since it does not depend on a particular partitioning of a network, but only on the fact that the network is to a certain extent modular.

Special emphasis should be placed on the methods introduced for the analysis of \emph{temporal} networks, i.e. systems where the occurrence of edges varies over time.
These systems are particularly intractable due to the importance of causality on paths.
We have found a method to obtain the accessibility graph of a temporal network.
We believe that the definition of accessibility of temporal networks contributes a key element for a theoretical framework for the macroscopic analysis of these systems, because it maps the whole causal path structure of the system onto a single mathematical object.
Moreover, we have introduced \emph{unfolding accessibility} as a novel formalism for the evaluation of shortest path durations in temporal networks.
This approach is able to reveal characteristic timescales for the traversal of temporal networks.
Knowledge of these timescales is of fundamental importance for the estimation of spreading times.

In addition to the above, the accessibility graph of a temporal network can be compared to its aggregated, static counterpart.
Using this concept, we have defined the novel measure of \emph{causal fidelity}, which quantifies the goodness of the static approximation of a temporal network from the causal point of view.
This measure is of major importance, since static network analysis provides a huge toolbox for quantifying network properties. 
Since there is no such toolbox for temporal networks yet, the static approximation of a temporal network can give useful insights into the real system.
On the other hand, temporal networks with low causal fidelities should be analyzed with care, when static network tools are used.
In particular, a low causal fidelity implies that disease outbreaks are systematically overestimated in the static approximation.


\paragraph{Outlook\color{Cayenne}{.}}
The generalization of the \emph{clustering coefficient} known from static networks (see equation \eqref{eq:clustering_coefficient}) can be done in a straightforward manner using the adjacency matrices of network snapshots.
Thus, the temporal clustering coefficient reads $C_{ijk}=\text{tr} (A_{i}A_{j}A_{k})/( \sum_{\mu,\nu\in\{i,j,k\}:\mu<\nu}\left[\sum _{\mu \nu }\left(A_{\mu}A_{\nu}\right)-\text{tr}\left(A_{\mu}A_{\nu}\right)\right])$.
The clustering coefficient has to be computed for all snapshot triples with indices $i<j<k$ and gives a 3-dimensional object.
This object can be contracted to a clustering matrix $\mat{C}$ with elements $c_{j-i,k-j}$ and a clustering vector $\mat{c}$ with elements $c_{k-i}$.
The former gives information about the node waiting times in closed triangles and the latter measures the total time of closed triangles in the network.

Although accessibility is a fundamental building block for the understanding of temporal networks, the development of a macroscopic theory of temporal networks is still in its infancy.
A promising approach would consist in mapping temporal network properties onto some static network image and analyze the latter instead.
Besides the obvious temporal nature of most network measures in temporal networks, the difficulty in such an approach lies in conceptual problems such as the degeneration of connected components.
These problems are mostly attributed to the non-transitivity of paths in temporal networks.
Hence, finding the transitive part of an accessibility graph could prove to be useful.
The author suggests to quantify \emph{transitivity} as follows: the transitivity matrix $\mathbf{T}=\mathcal{P}_T\circ \mathcal{P}_T ^2$ contains the transitive edges of the accessibility graph ($\circ $ denotes the Hadamard product).
This measure could help to identify transitive paths in temporal networks and facilitate the generalization of other concepts of static network analysis.





