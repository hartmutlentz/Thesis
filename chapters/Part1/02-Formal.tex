%\documentclass[openright,twoside,headsepline]{scrbook}
%\usepackage{graphicx}
%\usepackage[applemac]{inputenc}
%\usepackage{amsmath,amssymb,amsfonts,amsthm}
%\begin{document}

\chapter{Theory}

\section{Models of infectious diseases}

\subsection{Development}



\subsection{Infection dynamics}

The spread of infectious diseases can be modeled in terms of compartment models as described in section xx.
We differentiate between \emph{conceptional models} and \emph{realistic disease models}.
While the former class is used to provide conceptual results as for the computation of thresholds or to test theories \cite{Hethcote:2000}, realistic disease models use as many aspects as possible to provide a forecast of the spreading process.
Realistic disease models can be very complex and are beyond the scope of this work, thus the author focuses on the use of conceptional models.
The following section is inspired by the Lecture notes of J. R. Chasnov \cite{Chasnov:2010}.

\subsection{SI model}\label{sec:si_model}
Let us consider a population of $N$ individuals.
In the simplest case, the infection status of each individual is either susceptible or infected and there are no births and deaths on the population.
Susceptible individuals become infected, if they are in contact with an infected.
This mimics the behavior of an infectious disease without immunization, i.e. infected individuals stay permanently infected.

Provided that $\alpha $ is the rate, under which new susceptible become infected, the SI-model is as follows:
\begin{align}\label{eq:si_model}
\frac{dS}{dt} &= -\alpha SI \nonumber \\
\frac{dI}{dt} &= \alpha SI,
\end{align}
where $S$ and $I$ are the numbers of susceptible and infected individuals respectively.
The total population is $N=S+I$.
Thus, \eqref{eq:si_model} can be rewritten as
$$
\frac{dI}{dt}=\alpha (N-I)I,
$$
i.e. a logistic differential equation.
Hence, in the limit $t\rightarrow \infty $ the whole population is infected ($I(\infty )=N$). 

\subsection{SIR model}\label{sec:sir_model}
In contrast of the infection dynamics introduced in the previous section, many epidemics allow for an immunization of the individuals.
Examples are measles or whooping cough \cite{grenfell:92} \cite{andersonmay:92}.
In this case, individuals recover from the disease after being infected for a certain time period.
The infection scheme has to be extended to susceptible-infected-recovered (SIR) as in the following infection model \cite{kermack:27}:
\begin{align}\label{eq:sir_model}
\frac{dS}{dt} &= -\alpha SI \nonumber \\
\frac{dI}{dt} &= \alpha SI -\gamma I \nonumber \\
\frac{dR}{dt} &= \gamma I
\end{align}
where $\alpha $ is the infection rate and $\gamma $ is the immunization or recovery rate.
There is no analytic solution for the system \eqref{eq:sir_model}, but some fundamental conclusions can be obtained analytically.
We show a typical solution of \eqref{eq:sir_model} in figure \ref{fig:std_sir_model}.
\begin{figure}[htbp]
\begin{center}
\includegraphics{images/sir-sol.pdf}
\caption{Typical solution of a susceptible-infected-recovered (SIR) model. Parameters: $\alpha = 3$, $\gamma = 1$, $N=300$, $S_0=1$.}
\label{fig:std_sir_model}
\end{center}
\end{figure}


The SIR model shows more sophisticated features than the SI model \eqref{eq:si_model}.
To begin with, we analyze the fixed points of the system, i.e. $(S_*,I_*,R_*)$ where
\begin{equation}
\frac{dS_*}{dt} = -\alpha S_*I_* =0 ,\; \;\;
\frac{dI_*}{dt} = \alpha S_*I_* -\gamma I_* =0,\; \;\;
\frac{dR_*}{dt} = \gamma I_* = 0.
\end{equation}
It follows from the last equation that $I_*=0$ at the fixed point, where $S_*$ and $R_*$ can be arbitrary.
Hence, a fixed point is $(S_*,0,R_*)$.

Let us analyze the stability of the fixed point in the early phase of an infection.
Almost all individuals are susceptible and consequently $I_*=N-S_*$.
An outbreak occurs, if and only if $dI/dt >0$ in this phase, i.e.
\begin{equation}\label{eq:prelim_condition}
\frac{dI}{dt}=\alpha S_* (N-S_*) - \gamma (N-S_*)=(N-S_*)(\alpha S_* -\gamma ) >0.
\end{equation}
It follows from \eqref{eq:prelim_condition} that the number of infected grows, if
\begin{equation}\label{eq:prelim_rnod}
\alpha S_* / \gamma >1.
\end{equation}
Equation \eqref{eq:prelim_rnod} is extremely important in epidemiology, because it defines a threshold for the unfolding of an infection spreading process.
We call this fraction the \emph{basic reproduction number} $R_0$.
Recall that $S_* \approx N$ in the fixed point.
Thus it follows that the outbreak condition is
\begin{equation} \label{eq:r0}
R_0 = N \frac{\alpha }{\gamma } >1.
\end{equation}

The basic reproduction number describes the average number of follow-up infections by each infected individual.
It is one of the main goals in epidemiology to bring down the basic reproduction number of a disease below the critical value $R_0=1$.
This is the reason for the implementation of mass vaccination.
As one can immediately see from equation \eqref{eq:r0}, this can be done by reducing the infection rate $\alpha $ or by increasing the immunization rate $\gamma $.
In principle, one could also reduce the size of the initial population $S_*$.
As an example, reducing the infection rate can be done by increasing hygiene standards or appropriate behavior, say wearing warm clothes in winter time to avoid common cold.
The immunization rate can be increased by vaccination.

Let us now focus on the late phase of an SIR-infection.
In contrast to the SI-model of section \ref{sec:si_model} an SIR like outbreak does not necessarily infect the whole population, even if $R_0>1$.
The reason is that there has to be a critical mass of susceptible individuals in order to keep an infection alive (see equation \eqref{eq:prelim_rnod}).
The total number of infected during an infection given by the number of recovered at the end of the infection, since every recovered has to be in the infected state in the first place.
A central measure throughout this work is therefore the \emph{outbreak size} $R_\infty$.

To compute the outbreak size, we consider the second fixed point of \eqref{eq:sir_model}, i.e. the fixed point for $t \rightarrow \infty $.
At this point there are no infected and a fraction of the population is recovered.
Hence the fixed point is $(N-R_\infty , 0, R_\infty )$.
A simple way to obtain the outbreak size $R_\infty $ is to use equations \eqref{eq:sir_model} and compute 
$$
\frac{dS}{dR}=-\frac{\alpha }{\gamma } S 
$$
and separate the variables \cite{Chasnov:2010}.
This yields
$$
\int _{S_*} ^{N-R_\infty} \frac{dS}{S}=-\frac{\alpha }{\gamma } \int _{0} ^{R_\infty} dR ,
$$
which can be simplified to
$$
R_\infty = N-S_* e ^{-\frac{N\alpha}{\gamma}(R_\infty -R_*)}.
$$
This term is in general smaller than $N$, to that not all susceptible became infected during the spreading process.

Timescales:


\subsection{Epidemiological corrections}
Force of infection.



\section{Complex networks}



\section{Epidemics on networks}

%\end{document}
