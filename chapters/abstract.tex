\selectlanguage{english}
\begin{abstract}
The objective of this thesis is to examine the role of paths for the spread of infectious diseases on complex networks.
We demonstrate the importance of paths in the context of epidemiology for the case of static networks and then analyze paths in temporal networks.
As a central result, we introduce the \emph{unfolding accessibility} method, that allows for the analysis of the path structure of temporal networks.
%for temporal networks in order to analyze the path structure of these systems.

Provided that data about the contact structure for the spread of infectious diseases is known, a fundamental first step is the investigation of the \emph{domain} available for disease spread.
The structure of such a domain depends on the properties of the underlying network.
%Concerning the spread of infectious diseases on complex networks, detailed knowledge of the dynamical aspects of disease transmission is not known in most real-world scenarios.
%It turns out however that the pure topology of contact patterns is of major importance in this context.
%In contrast to the infection parameters of most diseases, the contact structures can be measured in great detail for a large number of real-world systems.
%Although these contact structures form complex networks, which exhibit a variety of properties in most examples, it turns out that the structure of paths in these systems defines the domain for any spreading process.
%On the one hand, the range of a node defines an upper limit for the size of disease outbreaks.
%On the other hand, the path structure of the whole system can be mapped onto the accessibility of the network.

In this thesis, we analyze the impact of two particular attributes of \emph{static} complex networks on the properties of their path structure.
As a case study, we analyze the properties of a livestock trade network in Germany in detail.
Among other features, this network exhibits a giant component and a significant modular structure.
We analyze the role of the giant component and the modular structure systematically.
The main findings here are first, networks close to the percolation threshold are likely to show two disjoint risk classes for the nodes and second, a modular structure causes a significant delay for disease outbreaks.

Furthermore, special emphasis should be placed on the methods introduced in this thesis for the analysis of \emph{temporal} networks, i.e. systems where the occurrence of edges varies over time.
The analysis of these systems is particularly challenging due to the importance of preserving causality on paths.
In this work we introduce a novel method to obtain the causal accessibility graph of a temporal network.
We are convinced that the definition of accessibility of temporal networks contributes a key element for a theoretical framework for the macroscopic analysis of these systems, because it maps the whole causal path structure of a system onto a single mathematical object.
Moreover, we introduce \emph{unfolding accessibility} as a novel formalism for the evaluation of shortest path durations in temporal networks.
This approach is able to reveal characteristic timescales for the traversal of temporal networks.
Knowledge of these timescales is of fundamental importance for the estimation of times needed for the spread of e.g. infectious diseases, rumors, information or goods.

Finally, the accessibility graph of a temporal network can be compared to its aggregated, static counterpart.
Using this concept, we define the novel measure of \emph{causal fidelity}, which quantifies the goodness of the static approximation of a temporal network from the causal point of view.

\paragraph{Keywords\color{Cayenne}{:}} Complex Network, Epidemiology, Temporal Network 
\end{abstract}

\cleardoublepage


\selectlanguage{ngerman}
\begin{abstract}
Ziel dieser Arbeit ist es, die Rolle von Pfaden f�r die Ausbreitung von Infektionskrankheiten auf komplexen Netzwerken zu untersuchen.
Zun�chst zeigen wir die Relevanz von Pfaden im Kontext der Epidemiologie in statischen Netzwerken und analysieren dann Pfade in zeitabh�ngigen Netzwerken.
Ein zentrales Ergebnis ist hierbei die \emph{unfolding accessibility} Methode, die eine Analyse der Pfadstruktur zeitabh�ngiger Netzwerke erlaubt.
Insofern Daten �ber die f�r die Verbreitung von Infektionskrankheiten relevanten Kontaktstrukturen verf�gbar sind, besteht ein fundamentaler erster Schritt in der Untersuchung des Gebietes (engl. \emph{domain}), auf dem �berhaupt eine Infektion stattfinden kann.
Die Struktur dieses Gebietes h�ngt von den Eigenschaften des zugrundeliegenden Netzwerkes ab.

In dieser Dissertation wird der Einfluss zweier bestimmter Merkmale \emph{statischer} komplexer Netzwerke auf die Eigenschaften ihrer Pfadstruktur untersucht.
Als Fallbeispiel analysieren wir hierf�r ein Viehhandelsnetzwerk in Deutschland im Detail.
Neben anderen Eigenschaften besitzt dieses Netzwerk eine Riesenkomponente und eine signifikante modulare Struktur.
Die Rolle der Riesenkomponente und der Modulstruktur werden systematisch untersucht.
Die wichtigsten Ergebnisse sind hierbei erstens, dass Netzwerke, die nahe an der Perkolationsschwelle liegen, mit gro�er Wahrscheinlichkeit zwei disjunkte Risikoklassen f�r Knoten aufweisen und zweitens, dass eine modulare Struktur eine signifikante Verz�gerung von Krankheitsausbr�chen zur Folge hat.

Hervorzuheben sind au�erdem die Methoden, die hier zur Analyse \emph{zeitabh�ngiger} Netzwerke vorgestellt werden.
Das sind Systeme, in denen das Auftreten von Kanten mit der Zeit variiert.
Die Analyse solcher Systeme ist besonders anspruchsvoll, weil Kausalit�t auf allen Pfaden gew�hrleistet sein muss.
In dieser Arbeit stellen wir eine neue Methode vor, mit der die kausale Erreichbarkeit (engl. \emph{accessibility}) eines zeitabh�ngigen Netzwerks als Graph berechnet werden kann.
Wir sind �berzeugt, dass die Definition der Erreichbarkeit ein Schl�sselelement f�r eine Theoretische Behandlung zeitabh�ngiger Netzwerke liefert, weil sie die gesamte kausale Struktur eines Systems auf ein einziges mathematisches Objekt abbildet.
Dar�ber hinaus stellen wir \emph{unfolding accessibility} als eine neue Methode zur Berechnung k�rzester Pfad-Dauern in zeitabh�ngigen Netzwerken vor.
Diese Herangehensweise erm�glicht es, charakteristische Zeitskalen f�r das Durchqueren von zeitabh�ngigen Netzwerken aufzuzeigen.
Die Kenntnis solcher Zeitskalen ist von fundamentaler Wichtigkeit f�r die Absch�tzung von Zeiten, die f�r die Verbreitung von z.B. Epidemien, Ger�chten, Information oder Waren ben�tigt werden.

Zu guter Letzt kann die Erreichbarkeit eines zeitabh�ngigen Netzwerks mit ihrem aggregierten, statischen Gegenst�ck verglichen werden.
Wir benutzen diesen Ansatz und definieren das neue Ma� der kausalen G�te (engl. \emph{causal fidelity}), die die G�te einer statischen Approximation eines zeitabh�ngigen Netzwerks quantifiziert.

%\vspace{-0.2cm}
%\vfill
%\noindent \textbf{Schlagw�rter}\color{Cayenne}{:} Netzwerk, Epidemiologie, zeitabh�ngiges Netzwerk
\paragraph{Schlagw�rter\color{Cayenne}{:}} Komplexes Netzwerk, Epidemiologie, zeitabh�ngiges Netzwerk 
\end{abstract}

% Back to main language
\selectlanguage{english}
\cleardoublepage