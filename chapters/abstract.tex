\selectlanguage{english}
\begin{abstract}
The outbreak pattern of any infectious disease is limited by the topology of the underlying substrate.
For several years, detailed information about various examples of such substrates in the form of complex networks are available.
This information describes for instance interpersonal contact structures (social networks), the travel behavior of people (travel) or animals (migration) or the externally controlled, industrial transfer of livestock (livestock trade).
Networks are described by nodes (actors), which are connected via edges (links).
Although the nature of the above networks may be very different, it has been found that many networks share similar properties.
In this work, the influence of some of these properties is analyzed for the spread of infectious diseases.
As a case example, the network of German pork trade is examined in detail.
The findings are generalized to other networks where possible.

For the spread of infectious diseases, the path is a central concept, because the distribution of paths in the network limits the course of an outbreak.
The severity of paths and their lengths is directly linked to the connected components of the network.
A connected component is a sub-network, on which all nodes can reach each other.
The components of a network can be considered as isolated from one another and thus form epidemiological islands.
The livestock trade network shows a typical component distribution networks, which implies that the nodes form two risk classes.
In addition to the components, other subnetworks are determined for the livestock trade network:
So called modules are sub-networks that are internally densely connected and contain only few edges to other modules.
Thus modules correspond to a less severe type of islands in the network.
In this work the influence of components and modules on disease outbreaks is studied systematically.
It turns out that the existence of a modular structure can cause a significant delay of a disease outbreak.

In the last part of the paper the concept of paths for time-dependent networks studied.
These are characterized by the fact that the edges fluctuate over time.
Such behavior can not be described by means of classical network analysis.
In particular, the causality of paths plays a crucial role here.
As a particular result of this work, a new method for the calculation of paths between all pairs of nodes should be emphasized.
The method includes the step by step construction of paths and is therefore called unfolding accessibility.
Using this method, we can detect characteristic time scales and quantify the goodness of a static approximation of the time-dependent system.
Further, the method can be used to compute the intensity of mixing in the network.
\paragraph{Keywords\color{Cayenne}{:}} Complex Network, Epidemiology, Temporal Network 
\end{abstract}

\cleardoublepage


\selectlanguage{ngerman}
\begin{abstract}
Das Ausbruchsmuster jedes infekti�sen Krankheitsausbruchs ist durch die Topologie des darunter liegenden Substrates beschr�nkt.
Seit einigen Jahren sind detaillierte Informationen �ber verschiedenste Beispiele solche Substrate in Form von komplexen Netzwerken verf�gbar.
Diese beschreiben z.B. zwischenmenschliche Kontaktstrukturen (soziale Netzwerke), das Reiseverhalten von Menschen (Travel) bzw. Tieren (Migration) oder die extern gesteuerte, industrielle Verbringung von Nutztieren (Viehhandel).
Netzwerke werden durch Knoten (Akteure) beschrieben, die �ber Kanten (Verbindungen) miteinander verbunden sind.
Obwohl die Natur der oben genannten Netzwerke sehr verschieden sein kann, hat sich herausgestellt, dass viele Netzwerke �hnliche Eigenschaften teilen.
In dieser Arbeit wird der Einfluss einiger dieser Eigenschaften auf die Ausbreitung von Infektionskrankheiten untersucht.
Als Fallbeispiel wird das Netzwerk des deutschen Schweinehandels n�her untersucht.
An entsprechenden Stellen werden jedoch (wenn m�glich) die gewonnenen Erkenntnisse verallgemeinert.

F�r die Verbreitung von Infektionskrankheiten ist der Pfad ein zentraler Begriff, weil die Verteilung der Pfade im Netzwerk den Verlauf eines Ausbruchs begrenzt.
Die Auspr�gung von Pfaden und deren L�ngen ist unmittelbar an die zusammenh�ngenden Komponenten des Netzwerks gekoppelt.
Eine zusammenh�ngende Komponente ist ein Subnetzwerk, auf dem sich alle Knoten gegenseitig erreichen k�nnen.
Verschiedene Komponenten eines Netzwerks k�nnen als isoliert voneinander betrachtet werden und bilden damit epidemiologische Inseln.
Das Viehhandelsnetzwerk zeigt eine f�r Netzwerke typische Komponentenverteilung, die dazu f�hrt, dass die Knoten zwei Risikoklassen bilden.
Neben den Komponenten werden f�r das Viehhandelsnetzwerk auch andere Subnetzwerke bestimmt:
Sog. Module sind Subnetzwerke, die intern stark vernetzt sind und nur wenige Kanten zu anderen Modulen enthalten.
Damit entsprechen Module einer weniger strengen Art von Inseln im Netzwerk.
In dieser Arbeit wird der Einfluss von Komponenten und Modulen auf Krankheitsausbr�che systematisch untersucht.
Es stellt sich heraus, dass die Existenz einer modularen Struktur zu einer signifikanten Verlangsamung eines Krankheitsausbruchs f�hren kann.

Im letzten Teil der Arbeit das Konzept der Pfade f�r zeitabh�ngige Netzwerke untersucht.
Diese zeichnen sich dadurch aus, dass die Kanten �ber die Zeit fluktuieren.
Ein solches Verhalten kann nicht mit den Mitteln der klassischen Netzwerkanalyse beschrieben werden.
Insbesondere spielt die Kausalit�t von Pfaden hier eine entscheidende Rolle.
Als besonderes Ergebnis dieser Arbeit ist eine neue Methode f�r die Berechnung der Pfade zwischen allen Knotenpaaren im Netzwerk hervorzuheben.
Die Methode ber�cksichtigt die Schrittweise Konstruktion von Pfaden und wird deshalb als hei�t Pfadentfaltung (unfolding accessibility) bezeichnet.
Mit Hilfe dieser Methode lassen sich charakteristische Zeitskalen detektieren und die G�te einer statischen Approximation des zeitabh�ngigen Systems berechnen.
Ferner kann die Methode benutzt werden, um die Auspr�gung bestimmter Mischungseigenschaften im Netzwerk messbar zu machen.
\vspace{-0.2cm}
%\vfill
%\noindent \textbf{Schlagw�rter}\color{Cayenne}{:} Netzwerk, Epidemiologie, zeitabh�ngiges Netzwerk
\paragraph{Schlagw�rter\color{Cayenne}{:}} Netzwerk, Epidemiologie, zeitabh�ngiges Netzwerk 
\end{abstract}

% Back to main language
\selectlanguage{english}
\cleardoublepage