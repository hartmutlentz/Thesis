\selectlanguage{english}
\begin{abstract}
The objective of this thesis is to examine the role of paths for the spread of infectious diseases on networks.
We demonstrate the importance of paths in the context of epidemiology for the case of static networks and then carry over the approach to temporal networks.
As a central result, we introduce the unfolding accessibility method for temporal networks in order to analyze the path structure of these systems.

Concerning the spread of infectious diseases on complex networks, detailed knowledge of the dynamical aspects of disease transmission is not known in most real-world scenarios.
It turns out however that the pure topology of contact patterns is of major importance in this context.
In contrast to the infection parameters of most diseases, the contact structures can be measured to great detail for a large number of real-world systems.
Although these contact structures form complex networks, which exhibit a variety of properties in most examples, it turns out that the structure of paths in these systems defines the domain for any spreading process.
On the one hand, the range (out-component) of a node defines an upper limit for the size of disease outbreaks.
On the other hand, the path structure of the whole system can be mapped onto the accessibility (transitive closure) of the network.

In this thesis, we analyze the impact of two particular attributes of \emph{static} complex networks on the properties of their paths.
As a case study, we analyze the properties of a livestock trade network in Germany in detail.
Among other features, this network exhibits a giant component and a significant modular structure.
We analyze the role of the giant component and the modular structure systematically.
The main findings here are that first percolating networks are likely to show two disjoint risk classes for the nodes, and second that a modular structure causes a significant delay for disease outbreaks.

Special emphasis should be placed on the methods introduced for the analysis of \emph{temporal} networks, i.e. systems where the occurrence of edges varies over time.
These systems are particularly intractable due to the importance of causality on paths.
We show a novel method to obtain the accessibility graph of a temporal network.
We believe that the definition of accessibility of temporal networks contributes a key element for a theoretical framework for the macroscopic analysis of these systems, because it maps the whole causal path structure of the system onto a single mathematical object.
Moreover, we introduce \emph{unfolding accessibility} as a novel formalism for the evaluation of shortest path durations in temporal networks.
This approach is able to reveal characteristic timescales for the traversal of temporal networks.
Knowledge of these timescales is of fundamental importance for the estimation of spreading times.

In addition to the above, the accessibility graph of a temporal network can be compared to its aggregated, static counterpart.
Using this concept, we define the novel measure of \emph{causal fidelity}, which quantifies the goodness of the static approximation of a temporal network from the causal point of view.

\paragraph{Keywords\color{Cayenne}{:}} Complex Network, Epidemiology, Temporal Network 
\end{abstract}

\cleardoublepage


\selectlanguage{ngerman}
\begin{abstract}

%\vspace{-0.2cm}
%\vfill
%\noindent \textbf{Schlagw�rter}\color{Cayenne}{:} Netzwerk, Epidemiologie, zeitabh�ngiges Netzwerk
\paragraph{Schlagw�rter\color{Cayenne}{:}} Netzwerk, Epidemiologie, zeitabh�ngiges Netzwerk 
\end{abstract}

% Back to main language
\selectlanguage{english}
\cleardoublepage