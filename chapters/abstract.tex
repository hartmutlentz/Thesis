\selectlanguage{english}
\begin{abstract}
The objective of this thesis is to examine the role of paths for the spread of infectious diseases on complex networks. We demonstrate the importance of paths in the context of epidemiology for the case of static and temporal networks. As a central result, we introduce the unfolding accessibility method, that allows for the analysis of the path structure of temporal networks.

In this thesis, we analyze the impact of two particular attributes of static networks on the properties of their path structure. As a case study, we analyze the properties of a livestock trade network in Germany. This network exhibits a giant component and a modular structure. The main findings here are that networks close to the percolation threshold are likely to show two disjoint risk classes for the nodes and, a modular structure causes a significant delay for disease outbreaks.

Furthermore, special emphasis should be placed on the methods introduced in this thesis for the analysis of temporal networks, i.e. systems where the occurrence of edges varies over time. In this work we introduce a novel method to obtain the causal accessibility graph of a temporal network. Moreover, we introduce unfolding accessibility as a novel formalism for the evaluation of shortest path durations in temporal networks. This approach is able to reveal characteristic timescales for the traversal of temporal networks. Knowledge of these timescales is of fundamental importance for the estimation of times needed for the spread of infectious diseases.

The accessibility graph of a temporal network can be compared to its aggregated counterpart. Hence we define the causal fidelity, which quantifies the goodness of the static approximation of a temporal network from the causal point of view.

\paragraph{Keywords\color{Cayenne}{:}} Complex Network, Epidemiology, Temporal Network, Statistical Physics
\end{abstract}

\cleardoublepage


\selectlanguage{ngerman}
\begin{abstract}
Ziel dieser Arbeit ist es, die Rolle von Pfaden f�r die Ausbreitung von Infektionskrankheiten auf komplexen Netzwerken zu untersuchen. Wir zeigen die Relevanz von Pfaden im Kontext der Epidemiologie in statischen und zeitabh�ngigen Netzwerken. Ein zentrales Ergebnis ist hierbei die Erreichbarkeitsentwicklung, die eine Analyse der Pfadstruktur zeitabh�ngiger Netzwerke erlaubt.

In dieser Dissertation wird der Einfluss zweier bestimmter Merkmale statischer Netzwerke auf die Eigenschaften ihrer Pfadstruktur untersucht. Als Fallbeispiel analysieren wir hierf�r ein Viehhandelsnetzwerk in Deutschland. Dieses Netzwerk besitzt eine Riesenkomponente und eine modulare Struktur. Die wichtigsten Ergebnisse sind hierbei, dass Netzwerke, die nahe an der Perkolationsschwelle liegen, mit gro�er Wahrscheinlichkeit zwei disjunkte Risikoklassen f�r Knoten aufweisen und, dass eine modulare Struktur eine signifikante Verz�gerung von Krankheitsausbr�chen zur Folge hat.

Hervorzuheben sind au�erdem die Methoden, die hier zur Analyse zeitabh�ngiger Netzwerke vorgestellt werden. Das sind Systeme, in denen das Auftreten von Kanten mit der Zeit variiert.
In dieser Arbeit stellen wir eine neue Methode vor, mit der die kausale Erreichbarkeit eines zeitabh�ngigen Netzwerks berechnet werden kann.

Dar�ber hinaus stellen wir Erreichbarkeitsentwicklung als eine neue Methode zur Berechnung k�rzester Pfaddauern in zeitabh�ngigen Netzwerken vor. Diese Herangehensweise erm�glicht es, charakteristische Zeitskalen f�r das Durchqueren von zeitabh�ngigen Netzwerken aufzuzeigen.
Die Kenntnis solcher Zeitskalen ist von fundamentaler Wichtigkeit f�r die Absch�tzung von Zeiten, die f�r die Verbreitung von Epidemien ben�tigt werden.

Die Erreichbarkeit eines zeitabh�ngigen Netzwerks kann mit ihrem aggregierten Gegenst�ck verglichen werden. Damit definieren wir die Kausalit�tstreue, die die G�te einer statischen Approximation eines zeitabh�ngigen Netzwerks quantifiziert.

%\vspace{-0.2cm}
%\vfill
%\noindent \textbf{Schlagw�rter}\color{Cayenne}{:} Netzwerk, Epidemiologie, zeitabh�ngiges Netzwerk
\paragraph{Schlagw�rter\color{Cayenne}{:}} Komplexes Netzwerk, Epidemiologie, zeitabh�ngiges Netzwerk, Statistische Physik
\end{abstract}

% Back to main language
\selectlanguage{english}
\cleardoublepage